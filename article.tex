%%%%%%%%%%%%%%%%%%%%%%%%%%%%%%%%%%%%%%%%%%%%%%%%%%%%%%%%%%%%%%%%%%%%%%%%%%%%%%%

\documentclass[a4paper]{llncs}

%%%%%%%%%%%%%%%%%%%%%%%%%%%%%%%%%%%%%%%%%%%%%%%%%%%%%%%%%%%%%%%%%%%%%%%%%%%%%%%

\usepackage{amsfonts}
\usepackage{amsmath}
\usepackage[all]{xy}
\usepackage{algorithm}
\usepackage[noend]{algpseudocode}

%%%%%%%%%%%%%%%%%%%%%%%%%%%%%%%%%%%%%%%%%%%%%%%%%%%%%%%%%%%%%%%%%%%%%%%%%%%%%%%

%%% 
%%% complexity.tex
%%% 

\usepackage{xspace}

%%% ----------------------------------------------------------------------
%%% complexity classes
%%% ----------------------------------------------------------------------

% TIME
\newcommand{\DTIMEX}{{\sf\bf DTIME}}
\newcommand{\DTIMEclass}{\DTIMEX\xspace}
\newcommand{\DTIME}{\DTIMEclass}
% NL class
\newcommand{\NLclassbase}{{\sf\bf NL}}
\newcommand{\NLclass}{\NLclassbase\xspace}
% P class
\newcommand{\Pclassbase}{{\sf\bf P}}
\newcommand{\Pclass}{\Pclassbase\xspace}
% NP class
\newcommand{\NPclassbase}{{\sf\bf NP}}
\newcommand{\NPclass}{\NPclassbase\xspace}
% coNP class
\newcommand{\coNPclassbase}{{\sf\bf coNP}}
\newcommand{\coNPclass}{\coNPclassbase\xspace}
% PSPACE class
\newcommand{\PSPACEclassbase}{{\sf\bf PSPACE}}
\newcommand{\PSPACEclass}{\PSPACEclassbase\xspace}
% MAXSNP class
\newcommand{\MaxSNPclassbase}{{\sf\bf MaxSNP}}
\newcommand{\MaxSNPclass}{\MaxSNPclassbase\xspace}
% MAXNP class
\newcommand{\MaxNPclassbase}{{\sf\bf MaxNP}}
\newcommand{\MaxNPclass}{\MaxNPclassbase\xspace}
% EPTAS class
\newcommand{\EPTASclassbase}{{\sf\bf EPTAS}}
\newcommand{\EPTASclass}{\EPTASclassbase\xspace}
% FPTAS class
\newcommand{\FPTASclassbase}{{\sf\bf FPTAS}}
\newcommand{\FPTASclass}{\FPTASclassbase\xspace}
% PTAS class
\newcommand{\PTASclassbase}{{\sf\bf PTAS}}
\newcommand{\PTASclass}{\PTASclassbase\xspace}
% APX class
\newcommand{\APXclassbase}{{\sf\bf APX}}
\newcommand{\APXclass}{\APXclassbase\xspace}
% log-APX class
\newcommand{\logAPXclassbase}{{\sf\bf log{\tt -}APX}}
\newcommand{\logAPXclass}{\logAPXclassbase\xspace}
% poly-APX class
\newcommand{\polyAPXclassbase}{{\sf\bf poly{\tt -}APX}}
\newcommand{\polyAPXclass}{\polyAPXclassbase\xspace}
% exp-APX class
\newcommand{\expAPXclassbase}{{\sf\bf exp{\tt -}APX}}
\newcommand{\expAPXclass}{\expAPXclassbase\xspace}
% NPO class
\newcommand{\NPOclassbase}{{\sf\bf NPO}}
\newcommand{\NPOclass}{\NPOclassbase\xspace}
% #P class
\newcommand{\sharpPclassbase}{\#{\sf\bf P}}
\newcommand{\sharpPclass}{\sharpPclassbase\xspace}
% FPT class
\newcommand{\FPTclassbase}{{\sf\bf FPT}}
\newcommand{\FPTclass}{\FPTclassbase\xspace}
% W class
\newcommand{\Wclassbase}[1]{{\sf\bf W[#1]}}
\newcommand{\Wclass}[1]{\Wclassbase{#1}\xspace}
% W class
\newcommand{\XPclassbase}{{\sf\bf XP}}
\newcommand{\XPclass}{\XPclassbase\xspace}
% WNL class
\newcommand{\WNLclassbase}{{\sf\bf WNL}}
\newcommand{\WNLclass}{\WNLclassbase\xspace}
% ZPP class
\newcommand{\ZPPclassbase}{{\sf\bf ZPP}}
\newcommand{\ZPPclass}{\ZPPclassbase\xspace}
% NPK class
\newcommand{\NPKclassbase}{{\sf\bf NPK}}
\newcommand{\NPKclass}{\NPKclassbase\xspace}
\newcommand{\NPKandclass}{\text{$\NPKclass_\text{and}$}\xspace}
\newcommand{\NPKzeroandclass}{\text{$\NPKclass^0_\text{and}$}\xspace}
\newcommand{\NPKorclass}{\text{$\NPKclass_\text{or}$}\xspace}
\newcommand{\NPKzeroorclass}{\text{$\NPKclass^0_\text{or}$}\xspace}

%%% ----------------------------------------------------------------------
%%% complete
%%% ----------------------------------------------------------------------

% keyword
\newcommand{\complete}{\text{-complete}}
% NL-complete
\newcommand{\NLcomplete}{\NLclassbase\complete\xspace}
\newcommand{\NLC}{\NLcomplete}
% P-complete
\newcommand{\Pcomplete}{\Pclassbase\complete\xspace}
\newcommand{\PC}{\Pcomplete}
% NP-complete
\newcommand{\NPcomplete}{\NPclassbase\complete\xspace}
\newcommand{\NPC}{\NPcomplete}
% coNP-complete
\newcommand{\coNPcomplete}{\coNPclassbase\complete\xspace}
\newcommand{\coNPC}{\coNPcomplete}
% PSPACE-complete
\newcommand{\PSPACEcomplete}{\PSPACEclassbase\complete\xspace}
\newcommand{\PSPACEC}{\PSPACEcomplete}
% MAXSNP-complete
\newcommand{\MaxSNPcomplete}{\MaxSNPclassbase\complete\xspace}
\newcommand{\MaxSNPC}{\MaxSNPcomplete}
% APX-complete
\newcommand{\APXcomplete}{\APXclassbase\complete\xspace}
\newcommand{\APXC}{\APXcomplete}
% #P-complete
\newcommand{\sharpPcomplete}{\sharpPclassbase\complete\xspace}
\newcommand{\sharpPC}{\sharpPcomplete}
% W[i]-complete
\newcommand{\Wcomplete}[1]{\Wclassbase{#1}\complete\xspace}
\newcommand{\WC}[1]{\Wcomplete{#1}}
% WNL-complete
\newcommand{\WNLcomplete}{\WNLclassbase\complete\xspace}
\newcommand{\WNLC}{\WNLcomplete}

%%% ----------------------------------------------------------------------
%%% hard
%%% ----------------------------------------------------------------------

% keyword
\newcommand{\hard}{\text{-hard}}
% NL-hard
\newcommand{\NLhard}{\NLclassbase\hard\xspace}
\newcommand{\NLH}{\NLhard}
% P-hard
\newcommand{\Phard}{\NPclassbase\hard\xspace}
\newcommand{\PH}{\Phard}
% NP-hard
\newcommand{\NPhard}{\NPclassbase\hard\xspace}
\newcommand{\NPH}{\NPhard}
% coNP-hard
\newcommand{\coNPhard}{\coNPclassbase\hard\xspace}
\newcommand{\coNPH}{\coNPhard}
% PSPACE-hard
\newcommand{\PSPACEhard}{\PSPACEclassbase\hard\xspace}
\newcommand{\PSPACEH}{\PSPACEhard}
% MAXSNP-hard
\newcommand{\MaxSNPhard}{\MaxSNPclassbase\hard\xspace}
\newcommand{\MaxSNPH}{\MaxSNPhard}
% APX-hard
\newcommand{\APXhard}{\APXclassbase\hard\xspace}
\newcommand{\APXH}{\APXhard}
% WNL-hard
\newcommand{\WNLhard}{\WNLclassbase\hard\xspace}
\newcommand{\WNLH}{\WNLhard}
% #P-hard
\newcommand{\sharpPhard}{\sharpPclassbase\hard\xspace}
\newcommand{\sharpPH}{\sharpPhard}
% W[i]-hard
\newcommand{\Whard}[1]{\Wclassbase{#1}\hard\xspace}
\newcommand{\WH}[1]{\Whard{#1}}

%%% ----------------------------------------------------------------------
%%% hardness
%%% ----------------------------------------------------------------------

% keyword
\newcommand{\hardness}{\text{-hardness}}
% NP-hardness
\newcommand{\NPhardness}{\NPclassbase\hardness\xspace}
% APX-hardness
\newcommand{\APXhardness}{\APXclassbase\hardness\xspace}
% W[i]-hardness
\newcommand{\Whardness}[1]{\Wclassbase{#1}\hardness\xspace}
% WNL-hardness
\newcommand{\WNLhardness}{\WNLclassbase\hardness\xspace}

%%% ----------------------------------------------------------------------
%%% completeness
%%% ----------------------------------------------------------------------

% keyword
\newcommand{\completeness}{\text{-completeness}}
% NL-completeness
\newcommand{\NLcompleteness}{\NLclassbase\completeness\xspace}
% P-completeness
\newcommand{\Pcompleteness}{\NPclassbase\completeness\xspace}
% NP-completeness
\newcommand{\NPcompleteness}{\NPclassbase\completeness\xspace}
% APX-completeness
\newcommand{\APXcompleteness}{\APXclassbase\completeness\xspace}
% #P-completeness
\newcommand{\sharpPcompleteness}{\sharpPclassbase\completeness\xspace}
% W[i]-hard
\newcommand{\Wcompleteness}[1]{\W{#1}-\completeness\xspace}

%%% ----------------------------------------------------------------------
%%% reduction
%%% ----------------------------------------------------------------------

\newcommand{\reduction}{reduction}
\newcommand{\reductions}{reductions}
\newcommand{\reductible}{reductible}

\newcommand{\APTypeReduction}{AP}
\newcommand{\PTASTypeReduction}{PTAS}
\newcommand{\LTypeReduction}{L}
\newcommand{\ETypeReduction}{E}
\newcommand{\fptTypeReduction}{fpt}
\newcommand{\pptTypeReduction}{ptp}

% AP-reduction
\newcommand{\APreduction}{\APTypeReduction-\reduction\xspace}
\newcommand{\APreductions}{\APTypeReduction-\reductions\xspace}
\newcommand{\APreductible}{\APTypeReduction-\reductible\xspace}

% PTAS-reduction
\newcommand{\PTASeduction}{\PTASTypeReduction-\reduction\xspace}
\newcommand{\PTASreductions}{\PTASTypeReduction-\reductions\xspace}
\newcommand{\PTASreductible}{\PTASTypeReduction-\reductible\xspace}

% L-reduction
\newcommand{\Lreduction}{\LTypeReduction-\reduction\xspace}
\newcommand{\Lreductions}{\LTypeReduction-\reductions\xspace}
\newcommand{\Lreductible}{\LTypeReduction-\reductible\xspace}

% E-reduction
\newcommand{\Ereduction}{\ETypeReduction-\reduction\xspace}
\newcommand{\Ereductions}{\ETypeReduction-\reductions\xspace}
\newcommand{\Ereductible}{\ETypeReduction-\reductible\xspace}

% fpt-reduction
\newcommand{\fptreduction}{\fptTypeReduction-\reduction\xspace}
\newcommand{\fptreductions}{\fptTypeReduction-\reductions\xspace}
\newcommand{\fptreductible}{\fptTypeReduction-\reductible\xspace}

% ptp-reduction
\newcommand{\ptpreduction}{\ptpTypeReduction-\reduction\xspace}
\newcommand{\ptpreductions}{\ptpTypeReduction-\reductions\xspace}
\newcommand{\ptpreductible}{\ptpTypeReduction-\reductible\xspace}

% symbols
\DeclareMathOperator{\APreduce}{\text{$\leq_{\text{\APTypeReduction}}$}}
\DeclareMathOperator{\PTASreduce}{\text{$\leq_{\text{\PTASTypeReduction}}$}}
\DeclareMathOperator{\Lreduce}{\text{$\leq_{\text{\LTypeReduction}}$}}
\DeclareMathOperator{\Ereduce}{\text{$\leq_{\text{\ETypeReduction}}$}}
\DeclareMathOperator{\fptreduce}{\text{$\leq_{\text{\fptTypeReduction}}$}}
\DeclareMathOperator{\ptpreduce}{\text{$\leq_{\text{\fptTypeReduction}}$}}

%% 
%% Approximation
%%
\DeclareMathOperator{\poly}{poly}
\DeclareMathOperator{\POLY}{poly}
\DeclareMathOperator{\SIZE}{size}
\newcommand{\sol}{{\sf sol}\xspace}
\newcommand{\PB}[1]{\textsf{\scshape{#1}}}
\newcommand{\OPTname}{opt}
\newcommand{\OPT}{\text{$\mathsf{\bf \OPTname}$}}
\newcommand{\OPTpb}[1]{\text{$\mathsf{\OPTname}_{\PB{#1}}$}}
\newcommand{\ALGO}[1]{\text{\small{\ttfamily\sf #1}}}
\newcommand{\Approxname}{Approx}
\newcommand{\APPROX}[1]{\text{$\ALGO{\Approxname}_{\,\PB{#1}}$}}
\newcommand{\PCP}{{\sf\bf PCP}\xspace}

%%
%% Problem Definition
%%
\newcommand{\PbDef}[3]{%
\begin{center}
  \begin{tabular}{l}%
    \shadowbox{%
    \begin{minipage}[c]{.9\textwidth}
      \smallskip%
      \par\noindent%
      {#1}%
      \smallskip
      \par\noindent%
      $\bullet$
      \textbf{\textsf{Input}}~: #2% 
      \medskip
      \par\noindent%
      $\bullet$
      \textbf{\textsf{Question}}~:
      #3% 
      \smallskip%
      \par\noindent%
    \end{minipage}
  }% end shadowbox
  \end{tabular}%
\end{center}
}%
\newcommand{\PbDefinition}{\PbDef}

%%
%% Problem (Input + Output) Definition
%%
\newcommand{\PbInputOutputDef}[3]{%
\begin{center}
  \begin{tabular}{l}%
    \shadowbox{%
    \begin{minipage}[c]{.9\textwidth}
      \smallskip%
      \par\noindent%
      \PB{#1}%
      \medskip%
      \par\noindent%
      $\bullet$
      \textbf{\textsf{Input}}~: #2% 
      \medskip
      \par\noindent%
      $\bullet$
      \textbf{\textsf{Output}}~:
      #3% 
      \smallskip%
      \par\noindent%
    \end{minipage}
  }% end shadowbox
  \end{tabular}%
\end{center}
}%
\newcommand{\PbInputOutputDefinition}{\PbInputOutputDef}


%%
%% Optimization Problem Definition
%%

\newcommand{\OptPbDefinition}[4]{%
\begin{center}
  \begin{tabular}{l}%
    \shadowbox{%
    \begin{minipage}[c]{.9\textwidth}
      \par\noindent%
      \shadowbox{#1}%
      \par\noindent%
      $\bullet$
      \textbf{\textsf{Input}}~: #2% 
      \par\noindent%
      $\bullet$
      \textbf{\textsf{Solution}}~: #3%  
      \par\noindent%
      $\bullet$
      \textbf{\textsf{Measure}}~: #4% 
      \par\noindent%
    \end{minipage}
    }% end shadowbox
  \end{tabular}%
\end{center}
}%

%%
%% Parameterized Problem Definition
%%
\newcommand{\ParamPbDefinition}[4]{%
\begin{center}
  \begin{tabular}{l}%
    %\shadowbox{%
    \begin{minipage}[c]{.95\textwidth}
      % \smallskip%
      \par\noindent%
      #1%
      % \smallskip%
      \par\noindent%
      \textbf{\textsf{Input}}~: #2% 
      % \smallskip
      \par\noindent%
      \textbf{\textsf{Question}}~: #3%  
      % \smallskip
      \par\noindent%
      \textbf{\textsf{Parameter}}~: #4% 
      %\smallskip%
      \par\noindent%
    \end{minipage}
  %}% end shadowbox
  \end{tabular}%
\end{center}
}%
\newcommand{\ParamPbDefinitionTwo}[5]{%
\begin{center}
  \begin{tabular}{l}%
    \shadowbox{%
    \begin{minipage}[c]{.9\textwidth}
      \smallskip%
      \par\noindent%
      \shadowbox{#1}%
      \medskip%
      \par\noindent%
      $\bullet$
      \textbf{\textsf{Input}}~: #2% 
%      \medskip
      \par\noindent%
      $\bullet$
      \textbf{\textsf{Parameter}}~: #3% 
 %     \medskip
      \par\noindent%
      $\bullet$
      \textbf{\textsf{Parameter}}~: #4% 
      \medskip
      \par\noindent%
      $\bullet$
      \textbf{\textsf{Question}}~: #5%  
      \smallskip%
      \par\noindent%
    \end{minipage}
    }% end shadowbox
  \end{tabular}%
\end{center}
}%
\newcommand{\ParamPbDefinitionThree}[6]{%
\begin{center}
  \begin{tabular}{l}%
    \shadowbox{%
    \begin{minipage}[c]{.9\textwidth}
      \smallskip%
      \par\noindent%
      \shadowbox{#1}%
      \medskip%
      \par\noindent%
      $\bullet$
      \textbf{\textsf{Input}}~: #2% 
%      \medskip
      \par\noindent%
      $\bullet$
      \textbf{\textsf{Parameter}}~: #3% 
%      \medskip
      \par\noindent%
      $\bullet$
      \textbf{\textsf{Parameter}}~: #4% 
%      \medskip
      \par\noindent%
      $\bullet$
      \textbf{\textsf{Parameter}}~: #5%  
      \medskip
      \par\noindent%
      $\bullet$
      \textbf{\textsf{Question}}~: #6%  
      \smallskip%
      \par\noindent%
    \end{minipage}
    }% end shadowbox
  \end{tabular}%
\end{center}
}%
\newcommand{\ParamPbDefinitionFour}[7]{%
\begin{center}
  \begin{tabular}{l}%
    \shadowbox{%
    \begin{minipage}[c]{.9\textwidth}
      \smallskip%
      \par\noindent%
      \shadowbox{#1}%
      \medskip%
      \par\noindent%
      $\bullet$
      \textbf{\textsf{Input}}~: #2% 
%      \medskip
      \par\noindent%
      $\bullet$
      \textbf{\textsf{Parameter}}~: #3% 
%      \medskip
      \par\noindent%
      $\bullet$
      \textbf{\textsf{Parameter}}~: #4% 
%      \medskip
      \par\noindent%
      $\bullet$
      \textbf{\textsf{Parameter}}~: #5%  
%      \medskip
      \par\noindent%
    $\bullet$
      \textbf{\textsf{Parameter}}~: #6%  
      \medskip
      \par\noindent%
      $\bullet$
      \textbf{\textsf{Question}}~: #7%  
      \smallskip%
      \par\noindent%
    \end{minipage}
    }% end shadowbox
  \end{tabular}%
\end{center}
}%
\newcommand{\ParamPbDefinitionFive}[8]{%
\begin{center}
  \begin{tabular}{l}%
    \shadowbox{%
    \begin{minipage}[c]{.9\textwidth}
      \smallskip%
      \par\noindent%
      \shadowbox{#1}%
      \medskip%
      \par\noindent%
      $\bullet$
      \textbf{\textsf{Input}}~: #2% 
%      \medskip
      \par\noindent%
      $\bullet$
      \textbf{\textsf{Parameter}}~: #3% 
%      \medskip
      \par\noindent%
      $\bullet$
      \textbf{\textsf{Parameter}}~: #4% 
%      \medskip
      \par\noindent%
      $\bullet$
      \textbf{\textsf{Parameter}}~: #5%  
%      \medskip
      \par\noindent%
    $\bullet$
      \textbf{\textsf{Parameter}}~: #6%  
%      \medskip
      \par\noindent%
    $\bullet$
      \textbf{\textsf{Parameter}}~: #7%  
      \medskip
      \par\noindent%
      $\bullet$
      \textbf{\textsf{Question}}~: #8%  
      \smallskip%
      \par\noindent%
    \end{minipage}
    }% end shadowbox
  \end{tabular}%
\end{center}
}%
\newcommand{\ParamPbDefinitionSix}[9]{%
\begin{center}
  \begin{tabular}{l}%
    \shadowbox{%
    \begin{minipage}[c]{.9\textwidth}
      \smallskip%
      \par\noindent%
      \shadowbox{#1}%
      \medskip%
      \par\noindent%
      $\bullet$
      \textbf{\textsf{Input}}~: #2% 
%      \medskip
      \par\noindent%
      $\bullet$
      \textbf{\textsf{Parameter}}~: #3% 
%      \medskip
      \par\noindent%
      $\bullet$
      \textbf{\textsf{Parameter}}~: #4% 
%      \medskip
      \par\noindent%
      $\bullet$
      \textbf{\textsf{Parameter}}~: #5%  
%      \medskip
      \par\noindent%
      $\bullet$
      \textbf{\textsf{Parameter}}~: #6%  
%      \medskip
      \par\noindent%
      $\bullet$
      \textbf{\textsf{Parameter}}~: #7%  
%      \medskip
      \par\noindent%
      $\bullet$
      \textbf{\textsf{Parameter}}~: #8%  
      \medskip
      \par\noindent%
      $\bullet$
      \textbf{\textsf{Question}}~: #9%  
      \smallskip%
      \par\noindent%
    \end{minipage}
    }% end shadowbox
  \end{tabular}%
\end{center}
}%




%%%%%%%%%%%%%%%%%%%%%%%%%%%%%%%%%%%%%%%%%%%%%%%%%%%%%%%%%%%%%%%%%%%%%%%%%%%%%%%

\DeclareMathOperator{\RUN}{run}
\DeclareMathOperator{\RED}{red}
\DeclareMathOperator{\AV}{Av}

%%%%%%%%%%%%%%%%%%%%%%%%%%%%%%%%%%%%%%%%%%%%%%%%%%%%%%%%%%%%%%%%%%%%%%%%%%%%%%%

% 231,213
\newcommand{\myPattern}{231,231}

% avoid set
\DeclareMathOperator{\Avd}{Av}
\newcommand\Av[2]{\Avd_{{#1}}({#2})}

% le set de toutes les permitaton
\newcommand{\Perm}[1]{\mathcal{S}_{#1}}

% permutation text
\newcommand{\ptext}{\pi}

% permutation motif
\newcommand{\pmotif}{\sigma}

\newcommand\BV[2]{\genfrac{}{}{0pt}{}{#1}{#2}}

%Stripe
\DeclareMathOperator{\stripea}{s}
\newcommand{\stripe}[2]{\stripea_{{#1}}[{#2}]}
\newcommand{\stripew}[1]{\stripea_{{#1}}}

% compteur pour les proposition, definition etc...
\newcounter{num}
\setcounter{num}{0}
\newcommand{\num}{\stepcounter{num} }
\newcommand{\numl}[1]{\refstepcounter{num}\label{#1}}

%downstep
\newcommand{\dstep}{d}

%upstep
\newcommand{\ustep}{a}

%first
\DeclareMathOperator{\firsta}{L}
\newcommand{\first}[1]{\firsta({#1})}

\DeclareMathOperator{\factora}{F}
\newcommand{\factor}[1]{\factora({#1})}

%binvincular
\newcommand{\x}{X}
\newcommand{\y}{Y}
\newcommand{\bmotif}{(\sigma,\x,\y)}

\newcommand{\pbmotif}{\pmotif^+}
\newcommand{\pbmotifX}{\x_{\pbmotif}}
\newcommand{\pbmotifY}{\y_{\pbmotif}}

%Pattern Marching

\DeclareMathOperator{\PMa}{PM}
\newcommand{\PM}[6]{\PMa_{{#1}}^{{#2},{#3},{#4}}({#5},{#6})}


%LSC
\DeclareMathOperator{\LCSa}{LCS}
\newcommand{\LCS}[8]{\LCSa_{{#1},{#2},{#3}}^{{#4},{#5},{#6}}({#7},{#8})}

\DeclareMathOperator{\matcha}{M}
\newcommand{\match}[8]{\matcha_{{#1},{#2},{#3}}^{{#4},{#5},{#6}}({#7},{#8})}

%%%%%%%%%%%%%%%%%%%%%%%%%%%%%%%%%%%%%%%%%%%%%%%%%%%%%%%%%%%%%%%%%%%%%%%%%%%%%%%%

\begin{document}

%%%%%%%%%%%%%%%%%%%%%%%%%%%%%%%%%%%%%%%%%%%%%%%%%%%%%%%%%%%%%%%%%%%%%%%%%%%%%%%%

\title{Pattern matching for\\ $(213,231)$-avoiding permutations}
\author{Both Emerite NEOU, Romeo RIZZI}
\date{}

\author{%
	Both Emerite Neou\thanks{On a Co-tutelle  Agreement with the Department of Mathematics of the University of Trento}\inst{1}
  Romeo Rizzi\inst{2} \and
  St\'ephane Vialette\inst{1}
}% end author
\institute{%
	Universit\'e Paris-Est, LIGM (UMR 8049), CNRS, UPEM, ESIEE Paris, ENPC,
	F-77454, Marne-la-Vallée, France\\
  \email{\{neou,vialette\}@univ-mY.fr}
  \and
  Department of Computer Science,
  Università degli Studi di Verona, Italy \\
  \email{romeo.rizzi@univr.it}
}% end institute

\date{\today}

\maketitle

\begin{abstract}
	Given permutations $\sigma \in S_k$ and $\pi \in S_n$, the
  \emph{pattern containment} problem is to decide whether $\pi$ contains
  $\sigma$ as an order-isomorphic subsequence.
	We give a linear-time algorithm in case both $\pi$ and $\sigma$ avoid
	$213$ and $231$.
	For the special where only $\sigma$ avoids $213$ and $231$, we present a
	$O(kn^2)$ time algorithm. We extend our research to $(213,231)$-avoiding
	bivincular pattern and present a $O(kn^3)$ time algorithm.
\end{abstract}


%%%%%%%%%%%%%%%%%%%%%%%%%%%%%%%%%%%%%%%%%%%%%%%%%%%%%%%%%%%%%%%%%%%%%%%%%%%%%%%%

\section{Introduction}
\label{section:Introduction}

	A permutation $\pi$ is said to contain another permutation $\sigma$,
	in symbols $\sigma \preceq \pi$,
	if there exists a subsequence of entries of $\pi$ that has the same relative
	order as $\sigma$, and in this case $\sigma$ is said to be a
	\emph{pattern} of $\pi$.
	Otherwise, $\pi$ is said to \emph{avoid} the permutation $\sigma$.
	For example a permutation contains the pattern $123$ (resp. $321$) if it has
	an increasing (resp. decreasing) subsequence of length $3$.
	Here, note that members need not actually to be consecutive,
	merely increasing (resp. descreasing).
	As another example,
	$6152347$ contains $213$ but not $231$.
	During the last decade, the study of the pattern containment on permutations has
	become a very active area of research and
	a whole annual conference (\textsc{Permutation Patterns}) is now devoted
	to this topic.

	We consider here the so-called \emph{pattern containment} problem
	(also sometimes referred to as the \emph{pattern involvement problem}):
	Given two permutations $\sigma$ and $\pi$, this problem is to decide whether
	$\sigma \preceq \pi$ (the problem is ascribed to Wilf in \cite{Bose:Buss:Lubiw:1998}).
	The permutation containment problem is \NPhard~\cite{Bose:Buss:Lubiw:1998}.
	It is, however, polynomial-time solvable by brute-force enumeration
	if $\sigma$ has bounded size.
	Improvements to this algorithm were presented in
	\cite{Albert:Aldred:Atkinson:Holton:ISAAC:2001} and
	\cite{Ahal:Rabinovich:2008},
	the latter describing a $O(|\pi|^{0.47|\sigma|+o(|\sigma|)})$ time algorithm.
	Bruner and Lackner \cite{Bruner:Lackner:SWAT:2012}
	gave a fixed-parameter algorithm solving the pattern containment problem with
	an exponential worst-case runtime of $O(1.79^{\RUN(\pi)})$,
	where $\RUN(\pi)$ denotes the number of alternating runs of $\pi$.
	(This is an improvement upon the $O(2^{|\pi|})$ runtime required by
	brute-force search without imposing restrictions on $\sigma$ and $\pi$.)
	A recent major step was taken by􏰂 Marx and Guillemot
	\cite{Guillemot:Marx:SODA:2014}.
	They showed that
	the permutation containment problem is fixed-parameter tractable for
	parameter $|\sigma|$.

	A few particular cases of the pattern containment problem have been attacked successfully.
	The case of increasing patterns is solvable in
	$O(|\pi| \log \log |\sigma|)$ time in the RAM model \cite{Crochemore:Porat:2010},
	improving the previous 30-year bound of $O(|\pi| \log |\sigma|)$.
	(The algorithm also improves on the previous
	$O(|\pi| \log \log |\pi|)$ bound.)
	Furthermore, the patterns $132$, $213$, $231$, $312$ can all be handled in linear-time
	by stack sorting algorithms.
	Any pattern of length $4$ can be detected in $O(|\pi| \log |\pi|)$ time
	\cite{Albert:Aldred:Atkinson:Holton:ISAAC:2001}.
	Algorithmic issues for $321$-avoiding patterns containment has been investigated in
	\cite{Guillemot:Vialette:ISAAC:2009}.
	The pattern containment problem is also solvable in
	polynomial-time for separable patterns \cite{Ibarra:1997,Bose:Buss:Lubiw:1998}
	(see also \cite{Bouvel:Rossin:Vialette:CPM:2007} for LCS-like issues
	of separable permutations).
	Separable permutations are those permutations that contain neither
	$2413$ nor $3142$, and they are enumerated by the Schr{\"o}der numbers
	(sequence A006318 in OEIS).
	To the best of our knowledge,
	separable permutations first arose in the work of
	Avis and Newborn~\cite{Avis:Newborn:1981},
	who showed that they are precisely the permutations which can be sorted by an
	arbitrary number of pop-stacks in series,
	where a pop-stack is a restricted form of stack in which any pop operation
	pops all items at once.
	(Notice that the separable permutations include as a special case the
	stack-sortable permutations, which avoid the pattern $231$.)

	There exist many generalisation of patterns that are worth considering
	in the context of algorithmic issues in pattern involvement
	(see \cite{Kitaev:book:2011} for an up-to-date survey).
	\emph{Vincular patterns}, also called
	\emph{generalized patterns},
	resemble (classical) patterns, with the constraint that some of the letters in
	an occurrence must be consecutive.
	Of particular importance in our context,
	Bruner and Lackner \cite{Bruner:Lackner:SWAT:2012}
	proved that deciding whether a vincular pattern
	$\sigma$ of length $k$ occurs in a longer permutation
	$\pi$ is $W[1]$-complete for
	parameter $k$.
	\emph{Bivincular patterns} generalize classical patterns even further
	than vincular
	patterns. Indeed, in bivincular patterns,
	not only positions but also values
	of elements involved in a matching may be forced to be adjacent

	We focus in this paper on pattern matching issues for
	$(213,231)$-avoiding permutations.
	The number of $n$-permutations that avoid both
	$213$ and $231$ is
	$t_0 = 1$ for $n = 0$ and
	$t_n =2^{n-1}$ for $n\geq 1$ \cite{Simion:Schmidt:EJC:1985}.
	On an individual basis,
	the permutations that do not contain the permutation pattern $231$
	are exactly the \emph{stack-sortable permutations} and they are counted by
	the Catalan numbers \cite{Knuth:1997:ACP:260999}.
	A stack-sortable permutation is a permutation whose elements may be sorted by
	an algorithm whose internal storage is limited to a single stack data structure.
	As for $213$, it is well-known that
 	if $\pi = \pi_1\pi\,\ldots\,\pi_n$ avoids $132$, then its complement
 	$\pi' = (n+1-\pi_1)(n+1-\pi_2)\,\ldots\,(n+1-\pi_n)$ avoids $312$, and
 	the reverse of $\pi'$ avoids $213$.
 	From a combinatorial point of view,
 	B\'ona \cite{Bona:ElJC:2012}
 	showed the rather surprising fact that the cumulative number of
 	occurrences of the classical patterns $231$ and $213$ are the same on the
 	set of permutations avoiding $132$,
 	beside the pattern based statistics $231$ and $213$
 	do not have the same distribution on this set.
 	\emph{Almost avoidance} has also been considered;
 	a permutation $\pi$ almost avoids a set $X$ of permutations
 	if there is a way to remove a single element of $\pi$ to get a permutation
 	that avoids all elements in $P$
 	and $L_n(P)$ denotes the set of permutations of length $n$ that almost avoid
 	$P$.
 	It is shown in \cite{Griffiths:Smith:Warren:PMA:2011} that
 	$L_n(213, 231) = L_n(132, 231) = L_n(132, 312) = L_n(213, 312)$.

	This paper is organized as follows.

%	The permutation pattern problem asks if there exists a normalized subsequence of a text $\ptext$ which equals to a motif $\pmotif$.
%	If so we write $\pmotif \preceq \ptext$.
%	If not we say that $\ptext$ \textbf{avoid} $\pmotif$.
%	This was first introduced by Knuth \cite{Knuth:1997:ACP:260999}
%	who proves that permutation sortable by stacks
%	avoid the motif $231$.
%	The problem is NP-hard in general,
%	but for some classes of permutation,
%	polynomial time solution exists.
%	For instance the separable permutation \cite{Bose93patternmatching}.
%	We contribute studying this problem
%	and related one for the class of permutation avoiding (231,213).
%	This is interesting because this class is in linear bijection between the classes of (123,132),(132,213) and (132,213)
%	avoiding permutation, and all the classes obtainable by reversal or/and
%	complementation (see \cite{Simion1985383}).\\
%
%
%
%	This paper is organized as follows.
%	Section 2 presents the needed definition.
%	Section 3 solves the permutation pattern problem with
%	(213,231) avoiding permutations.
%	Section 4 focus on the problem the
%	related problem of computing a longest permutation avoiding (213,231).
%	Finally, section 5 summarizes
%	results for related problems.

%%%%%%%%%%%%%%%%%%%%%%%%%%%%%%%%%%%%%%%%%%%%%%%%%%%%%%%%%%%%%%%%%%%%%%%%%%%%%%%%

\section{Definition}
\label{section:Definitions}

A \emph{permutation} of length $n$ is a one-to-one function from an
$n$-element set to itself.
We write permutations as words
$\pi = \pi_1\pi_2\,\ldots\,\pi_n$, whose letters are distinct
and uasually consist of the integers $12\,\ldots\,n$.
We let
$\pi[i]$ stands for $\pi_i$.
For the sake of convenience, we let
$\pi[i:j]$ stand for
$\pi_i\pi_{i+1}\,\ldots\,\pi_j$,
$\pi[:j]$ stand for $\pi[1:j]$ and
$\pi[i:]$ stand for $\pi[i:n]$.
As usual, we let $S_n$ denote the set of all permutations of length $n$.

% The \emph{reduced form} of a permutation $\sigma$ on a set
% $\{j_1, j_2, \ldots, j_k\}$ where
% $j_1 < j_2 < \ldots < j_n$ is the permutation $\sigma'$
% obtained by renaming the letters of $\sigma$ so that
% $j_i$ is renamed $i$ for all $1 \leq i \leq k$.
% We let $\RED(\sigma)$ denote the reduced form of $\sigma$.
% For example $\RED(453) = 231$
% while $\RED(3174) = 2143$.

A permutation $\pi$ is said to \emph{contain} the permutation $\sigma$
if there exists a subsequence of (not necessarily consecutive)
entries of $\pi$ that has the same relative order as $\sigma$,
and in this case $\sigma$ is said to be a \emph{pattern} of
$\pi$, written $\sigma \preceq \pi$.
Otherwise, $\pi$ is said to \emph{avoid} the permutation $\sigma$.
For example, the permutation $\pi = 391867452$
contains the pattern $\sigma = 51342$,
as can be seen in the highlighted subsequence of
$\pi = 3\mathbf{9}\mathbf{1}8\mathbf{6}\mathbf{7}\mathbf{4}52$
(or
$\pi = 3\mathbf{9}\mathbf{1}8\mathbf{6}\mathbf{7}4\mathbf{5}2$
or
$\pi = 3\mathbf{9}\mathbf{1}8\mathbf{6}\mathbf{7}45\mathbf{2}$
).
Each subsequence $(91674$,
$91675$,
$91672$) is called a
\emph{copy},
\emph{instance}, or
\emph{occurrence} of $\sigma$.
Since the permutation $\pi = 391867452$  contains no increasing subsequence of length four, $\pi$ avoids $1234$.

Suppose $P$ is a set of permutations. We let $\AV_n(P)$ denote the
set of all $n$-permutations avoiding each permutation in $P$.
For the sake of convenience
(and as it is customary), we omit $P$'s braces thus having
e.g. $\AV_3(132,4312)$ instead of
$\AV_3(\{132,4312\})$.
A basic example is if
$\pi = \pi_1\pi_2\,\ldots\,\pi_n \in \AV_n(321)$, i.e.,
has no decreasing subsequence of length $3$, then its reverse,
$\pi' = \pi_n\pi_{n-1}\,\ldots\,\pi_1$ avoids $123$, i.e.,
has no increasing subsequence of length $3$.

An \emph{ascent} of a permutation $\pi \in S_n$ is any position
$1 \leq i < n$ where the following value is bigger than the current one.
That is, if $\pi = \pi_1\pi_2\,\ldots\,\pi_n$, then
$i$ is an ascent if $\pi_i < \pi_{i+1}$.
For example, the permutation
$345216$7 has ascents (at positions) $1,2,5$ and $6$.
Similarly, a \emph{descent} is a position
$1 \leq i < n$ with $\pi_i > \pi_{i+1}$,
so every $i$ with $1 \leq i < n$ either is an ascent or is a descent of
$\pi$.
Let $\ustep$ and $\dstep$ denote an ascend and a descend, respectively.
The \emph{stripe} $s_\pi$ of a permutation $\pi \in S_n$ is the word
$\stripe{\pi}{1} \stripe{\pi}{2} \ldots \stripe{\pi}{n-1} \in \{\ustep,\dstep\}^{n-1}$
defined by
$ \stripe{\pi}{i}= \ustep$ if $i$ an ascent in $\pi$ and
$\stripe{\pi}{i} = \dstep$ if $i$ a descent in $\pi$.
For example the stripe of the permutation
$\pi = 981234765$
is $\stripew{\pi} = \dstep\dstep\ustep\ustep\ustep\ustep\dstep\dstep$.
The stripe $\stripew{\sigma}$ is said to be a \emph{pattern} of the stripe $\stripew{\pi}$
(or $\stripew{\sigma}$ \emph{occurs} of the stripe $\stripew{\pi}$)
if $\stripew{\sigma}$ occurs in $\stripew{\pi}$ as a subsequence.

A \emph{bivincular pattern} (abbreviated BVP) $\sigma$
of length $k$ is a permutation in $S_k$ written in
two-line notation
(that is the top row is $12\,\ldots\,k$ and the bottom row
is a permutation $\sigma_1\sigma_2\,\ldots\,\sigma_k$).
We have the following conditions on the top and bottom rows
of $\sigma$:
\begin{itemize}
	\item
	If the bottom line of $\sigma$ contains
	$\underline{\sigma_i\sigma_{i+1}\,\ldots\,\sigma_j}$
	then the letters corresponding to
	$\sigma_i\sigma_{i+1}\,\ldots\,\sigma_j$ in an occurrence of
	$\sigma$ in a permutation must be adjacent, whereas there is
	no adjacency condition for
	non-underlined consecutive letters.
	Moreover if the bottom row of $\sigma$ begins with
	$_\llcorner{\sigma_1}$ then any occurrence of $\sigma$
	in a permutation $\pi$ must begin with the leftmost
	letter of $\pi$,
	and
	if the bottom row of $\sigma$ begins with
	${\sigma_k}_\lrcorner$ then any occurrence of $\sigma$
	in a permutation $\pi$ must end with the rightmost
	letter of $\pi$.
	\item
	If the top line of $\sigma$ contains
	$\overline{i\,i+1\,\ldots\,j}$ then the letters corresponding to
	$\sigma_i, \sigma_{i+1}, \ldots, \sigma_j$ in an
	occurrence of $\sigma$ in a permutation must be adjacent in values,
	whereas there is no value adjacency restriction for non-overlined
	letters.
	Moreover, if the top row of $\sigma$ begins with
	$^\ulcorner{1}$ then
	any occurrence of $\sigma$ is a permutation $\pi$ must begin with
	the smallest letter of $\pi$, and
	if top row of $\sigma$ ends with $k^\urcorner$ then
	any occurrence of $\sigma$ is a permutation $\pi$ must end with
	the largest letter of $\pi$.
\end{itemize}

For example,
let
$\sigma = \BV{\overline{123}}{\llcorner 213 \lrcorner}$.
If $\pi_i\pi_j\pi_\ell$ is an occurrence of $\sigma$ in $\pi \in S_n$,
then $\pi_i\pi_j\pi_k = (x+1)x(x+2)$ for some $1 \leq x \leq n-2$,
$i=1$ and $\ell = n$.
For example $316524$ contains one occurrence of $\sigma$ (the subsequence $324$),
whereas $42351$ avoids it.
It is straightforward to see that an $n$-permutation
$\pi$ avoids $\BV{\overline{123}}{\llcorner 213 \lrcorner}$ unless
$n \geq 3$,
$\pi_1 = x > 1$
and $\pi_n = x+1$
(in which case there is one occurrence of the pattern).

For algorithmic treatment, it will be convenient to have \ldots
Given a motif $\pmotif$ we represent the elements underlined by the set $\x$. An index $i$, $1 \leq i < |\sigma|$, of an element is in $\x$ if and only if $i$ and $i+1$ need to be adjacent ($\pmotif[i]$ and $\pmotif[i+1]$ are underlined). We also allow $0$ and $|\sigma|$ to be in the set $\x$ such that if $0 \in \x$ then we have $_\llcorner{\sigma_1}$ and if $|\sigma| \in \x$ then ${\sigma_{|\sigma|}}_\lrcorner$.
We represent the index that need to be overlined by the set $\y$. An index $i$ is in $\y$ if and only if $i$ and $i+1$ need to be adjacent ($i$ and $i+1$ are overlined). We also allow $0$ and $|\sigma|$ to be in the set $\y$ such that if $0 \in \y$ then we have $^\ulcorner{1}$ and if $|\sigma| \in \y$ then ${|\sigma|}^\urcorner$.

% A bivincular pattern avoids a set of permutations $P$ if its bottom
% row $\sigma_1\sigma_2\,\ldots\,\sigma$ avoids every permutation in $P$.

%%%%%%%%%%%%%%%%%%%%%%%%%%%%%%%%%%%%%%%%%%%%%%%%%%%%%%%%%%%%%%%%%%%%%%%%%%%%%%%%

\section{Both $\pi$ and $\sigma$ are $(213,231)$-avoiding}
\label{section:both are (231,213)-avoiding}

This section is devoted to presenting a fast algorithm for deciding
$\sigma \leq \pi$
in case both $\pi$ and $\sigma$ are $(213,231)$-avoiding.
We begin with an easy but crucial structure lemma.

\begin{lemma}[Folklore]
\label{lemma:first element is 1 or n}
The first element of any $(213,231)$-avoiding permutation
must be either $1$ or $n$.
\end{lemma}

\begin{proof}
Any other initial element would serve as a `$2$' in either a
$231$ or $213$ with $1$ and $n$ as the `$1$' and `$3$' respectively.
\end{proof}

Notice that Lemma~\ref{lemma:first element is 1 or n} gives as an
immediate consequence that the number of $(213,231)$-avoiding $n$-permutations
is $2^{n-1}$ for $n\geq 1$ \cite{Simion:Schmidt:EJC:1985}.
The above lemma also gains in interest in the form of the following corollary.

\begin{corollary}
\label{corollary:minmaxelement}
$\pi \in \AV_n(213,231)$ if and only if for $1 \leq i \leq n$,
$\pi[i:]$ start either with the maximal element
or the minimal element of $\pi[i:]$.
\end{corollary}

%\begin{corollary}
%
%If $\pi \in \Av{n}{(231,213)}$ its stripe can be define such as, \\
%$\stripe{\pi}{i} = \left\{
%	\begin{array}{l l}
%			\ustep & \quad \text{if $\pi[i]$ is the minimal element of $\pi[i:]$}\\
%			\dstep & \quad \text{if $\pi[i]$ is the maximal element of $\pi[i:]$}\\
%\end{array} \right.$
%\end{corollary}

The following lemma is central to all our algorithm.

\begin{lemma}
\label{lemma:MatchStripeToPermutation}
Let $\pi$ and $\sigma$ be two $(213,231)$-avoiding permutations,
and $t$ be a subsequence of $\pi$ of length $\sigma$.
Then $t$ is an occurrence of $\sigma$ in $\pi$ if and only if
$\stripew{t}$ is an occurrence of $\stripew{\sigma}$ in $\stripew{\pi}$.
\end{lemma}

\begin{proof}
	The forward direction is obvious.
	We prove the backward direction by induction on the size of the motif
	$\sigma$.
	The base case is a motif of size 2. Suppose that we have the motif $(12)$, thus the stripe of the motif is $\ustep$. Let's $t$ be a subsequence of $\ptext$ such that $\stripew{t}$ is a occurrence of $\stripew{\pmotif}$ in $\stripew{\ptext}$. Because we have $\stripe{t}{1}=\ustep$, the next element of $t[1]$ is $\ptext$ is superior to  $t[1]$, so we have a match. Same idea goes of motif $(21)$.
	Now, suppose that the lemma is true for every motif in $\Av{n}{(231,213)}$. Suppose that we have a subsequence $t$ of $\ptext$ such that $\stripew{t}$ is an occurrence of $\stripew{\pmotif}$ in $\stripew{\ptext}$, then $\stripew{t[2:]}$ is an occurrence of $\pmotif[2:]$ in $\ptext$. Moreover
	$\pmotif[2:] \in \Av{n}{(231,213)}$ so by hypothesis of induction $t[2:]$ is an occurrence of $\sigma[2:]$ in $\ptext$.
	Moreover if $\stripe{\pmotif}{1}$ is ascend (descent), then by lemma \ref{lemma:first element is 1 or n} $\pmotif[1]$ is the minimal (maximal) element of the motif. Furthermore $\stripew{t}$ is an occurrence of $\stripew{\pmotif}$ in $\stripew{\ptext}$ , so $\stripe{t}{1}$ is also ascend (descent) thus $t[1]$ is also the minimal (maximal) element of the subsequence $t$ which complete our proof.


\qed
\end{proof}

\begin{lemma}
\label{lemma:MatchStripeToPermutation}
Let $\pi$ and $\sigma$ be two $(213,231)$-avoiding permutations,
and $t$ be a subsequence of $\pi$ of length $|\sigma|$ such that $s_t = s_\sigma$.
Then, $t$ is an occurrence of $\sigma$ in $\pi$ if and only if
$s_t$ occurs as a subsequence in $s_\pi$.
\end{lemma}

\begin{proof}
  The forward direction is obvious.
  We prove the backward direction by induction on the size of the motif
  $\sigma$.
  The base case is a motif of size $2$.
  Suppose first that $\sigma = 12$ and hence $s_\sigma = a$.
  Let $t = \pi_{i_1}\pi_{i_2}$, $i_1 < i_2$, be a subsequence of $\pi$
  such that $s_t$ occurs as a subsequence in $s_\pi$.
  Since $s_\sigma = s_t = a$, this reduces to saying that
  $\pi_{i_1} < \pi_{i_2}$, and hence $t$ is an occurrence of $\sigma = 12$ in $\pi$.
  A similar argument shows that the lemma holds true for $\sigma = 21$.
  Now, assume that the lemma is true for all motifs up to size $k \geq 2$.
  Let $\sigma \in \Av{k+1}{231,213}$ and
  let $t = \pi_{i_1}\pi_{i_2}\,\ldots\,\pi_{i_{k+1}}$,
  $i_1 < i_2 < \ldots < i_{k+1}$,
  be a subsequence of $\pi$ of length $k+1$ such that
  $s_t = s_\sigma$.
  Suppose that $s_t$ occurs in $s_\pi$ as a subsequence.
  Then, $s_t[2:] = s_\sigma[2:]$ occurs as a subsequence in $s_\pi[2:]$.
  But $\sigma[2:] \in \Av{k}{231,213}$ and hence,
  by the inductive hypothesis, it follows that
  $t[2:]$ is an occurrence of $\sigma[2:]$ in $\pi[2:]$.
  Moreover, if $s_\sigma[1] = a$ (resp. $s_\sigma[1] = d$)
  then $\sigma[1]$ is the minimum (resp. maximum) element of $\sigma$
  (Lemma~\ref{lemma:first element is 1 or n}),
  and hence, since $s_t = s_\sigma$ occurs as a subsequence in $s_\pi$,
  $t[1]$ is the minimum (resp. maximum) element of $t$.
  Therefore, $t$ is an occurrence of $\sigma$ in $\pi$.
\qed
\end{proof}

We are now ready to solve the pattern containment problem in case
both $\pi$ and $\sigma$ are $(213, 231)$-avoiding.

\begin{proposition}
	Let $\pi$ and $\sigma$ be two $(213,231)$-avoiding permutations.
	One can decide whether $\pi$ contains $\sigma$ in linear-time.
\end{proposition}

\begin{proof}
According to Lemma~\ref{lemma:MatchStripeToPermutation} the problem reduces
to deciding whether $s_\sigma$ occurs as a subsequence in $s_\pi$.
A straightforward greedy approach solves this issue in linear-time.
Furthermore, in the light of Corollary~\ref{corollary:minmaxelement},
the effective construction of the two stripes is not required,
so it is an online algorithm.
\qed
% To solve the permutation pattern for a permutation in $\AV_n(213,231)$,
% we find a find a match of the stripe of the pattern in the stripe of the text.
% This can be done linearly by a greedy algorithm,
% by matching a step whenever it is possible.
% And thank to the definition of the definition of a stripe in
% Corollary~\ref{corollary:minmaxelement},
% we do not have to compute the stripe in advance.
% Therefore we have an linear (online) algorithm.
\end{proof}

%%%%%%%%%%%%%%%%%%%%%%%%%%%%%%%%%%%%%%%%%%%%%%%%%%%%%%%%%%%%%%%%%%%%%%%%%%%%%%%

\section{$\sigma$ only is $(231,213)$-avoiding}
\label{section:sigma only avoid 231 and 213}

This section focuses on the pattern containment problem
in case only the pattern $\sigma$ avoids $231$ and $213$.
We need to consider a specific decomposition $F$ of $\sigma$ into factors.
For $\sigma \in S_k$, the factorization
$\sigma = F(p) F(p-1) \ldots F(1)$ is defined as follows:
(i) $\sigma[k]$ is the last letter of $F(1)$,
(ii) for every $1 \leq i < k$,
if $s_\sigma[i] = s_\sigma[i+1]$ then
$\sigma[i]\sigma[i+1]$ is part of the same factor.
For every factor $F(j)$ of the decomposition, we let
$L(j)$ stand for the leftmost letter of $F(j)$.
For example,
for $\sigma = 981237654$,
we have
$\sigma = F(4) F(2) F(1)$ with
$F(1) =7654$, $F(2) = 123$ and $F(3) = 98$.
Furthermore,
$L(1) = 7$, $L(2) = 1$ and $L(3) = 9$.

\begin{proposition}
	\label{Proposition:sigma avoids 213 and 231}
	Let $\sigma \in \AV_k(213,231)$ and $\pi \in S_n$.
	One can decide in $O(kn^2)$ time
	and $O(kn^2)$ space if $\pi$ contains $\sigma$.
\end{proposition}

For the proof and the algorithm we introduce a decomposition : we decompose the stripe of the motif into factor of same letter, and label those factor from right to left. We also need to add the last index into the first factor. For each factor, we denote the left most element by $\firsta$. For example, given the motif $981237654$, $\factor{1} =7654$, $\first{1}=7$, $\factor{2}=123$, $\first{2}=1$, $\factor{3}=98$ and $\first{9}$.

\begin{proof}[of Proposition~\ref{Proposition:sigma avoids 213 and 231}]
Given an ascent (descent) factor $\factor{i}$ of a motif $\pmotif$, if we want to match the suffix of $\pmotif$ starting by $\factor{i}$, $\factor{i}\factor{i-1}\cdots\factor{1}$ and given that we know every match of $\factor{i-1}\cdots\factor{1}$, the optimal match is to choose the match of $\factor{i-1}\cdots\factor{1}$ that minimize (maximize) the maximal (minimal) element of the match of $\factor{i-1}\cdots\factor{1}$ and that allow a match of factor $\factor{i}$. In other word if we want to match $\pmotif[\first{i}:]$ with $\ptext[j:]$ such that $\first{i}$ is matched with element at index $j$ in $\ptext$, the optimal way of matching $\pmotif[\first{i-1}:]$ is to choose the minimal (maximal) element at index $j'$ of $\ptext[j:]$ such that : (1) $\pmotif[\first{i-1}:]$ is matched with $\ptext[j':]$ and element $\first{i-1}$ is matched with element at index $j'$, (2) the factor $\factor{i}$ is matched with $\ptext[j,j'-1]$ and element $\first{i}$ is matched with element at index $j$, (3) Every elements of the match of (2) if inferior (superior) to the minimal (maximal) element of the match of (1). Indeed when we will match $\factor{i+1}\factor{i}\factor{i-1}\cdots\factor{1}$, every element of $\factor{i+1}$ must be matched to element superior (inferior) to every element matched element of $\factor{i}\factor{i-1}\cdots\factor{1}$, particularly its must be superior (inferior) to the maximal (minimal) element of the match of $\factor{i}\factor{i-1}\cdots\factor{1}$.

\begin{remark}
Given an ascent (descent) factor $\factor{i}$, the maximal (minimal) element of $\factor{i}\factor{i-1}\cdots\factor{1}$ is $\first{i-1}$.
\end{remark}

\begin{remark}
An ascent (descent) factor 	$\factor{i}$ is a match in $\ptext[j:j']$  if and only if there exists an increasing (decreasing) subsequence in $\ptext[j:j']$ of size superior or equal to the size of $\factor{i}$.
\end{remark}

Consider the following problem :\\
$LM(i,j)$ = the optimal match of $\first{i-1}$
	if there exists a match of $\pmotif[\first{i}:]$
	with $\ptext[j:]$
	and element $\first{i}$
	is matched with element at index $j$.\\

This problem can be solve by induction.\\



\textbf{BASE :} \\
$
LM(1,j)=
\begin{cases}
		MIN_{j<j'}\{0\} \cup \{\ptext[j'] |\text{ $j'$ such that \indent } &
			\text{if $\stripe{\pmotif}{\first{1}}=\ustep$} \\
		\text{ $|\factor{1}| \leq LIS(j,j',\ptext[j'])  $}\}& \\
		&\\
		MAX_{j<j'}\{0\} \cup \{\ptext[j'] |\text{ $j'$ such that \indent } &
			\text{if $\stripe{\pmotif}{\first{1}}=\dstep$} \\
		\text{ $|\factor{1}| \leq LDS(j,j',\ptext[j']) $}\}&\\
\end{cases}
$

\textbf{STEP :} \\
$
LM(i,j)=
\begin{cases}
	MIN \{0\} \cup  I(i,j) &
	\text{if $\stripe{\pmotif}{\first{i}}=\ustep$}\\
	MAX \{0\} \cup  D(i,j) &
	\text{if $\stripe{\pmotif}{\first{i}}=\dstep$}\\
\end{cases}
$


With $I(i,j)$ ($D(i,j)$) the set of element such that
if  $j' \in I(i,j)$ then there exists
a match of $\pmotif[\first{i-1}:]$ in $\ptext[j':]$ and
the element $\first{i-1}$ is matched with element at index $j'$ and
there is a match of $\factor{i}$ in $\ptext[j:j'-1]$ and
the element $\first{i}$ is matched with element at index $j$ and every element is inferior (superior) to the match of
$\pmotif[\first{i-1}:]$ in $\ptext[j':]$.
Formally we define $I(s,i)$ and $D(s,i)$ such as:\\

$I(i,j)=\{j'|\text{$j<j'$ and $LM(i-1,j')\neq 0$}$ \\
\indent \indent \indent and  $|\factor{i}| \leq LIS(j,j'-1,\first{i-1})\}$ \\

$D(i,j)=\{j'|\text{$j<j'$ and $LM(i-1,j')\neq 0$}$ \\
\indent \indent \indent and  $|\factor{i}| \leq LDS(j,j'-1,\first{i-1}) \}$ \\

With LIS(i,j,k) (LDS(i,j,k)) is the longest increasing (decreasing) sequence in $\ptext$ starting at $i$ and ending at $j$,
with every element of this sequence
inferior (superior) to $k$.
LIS and LDS can be computed in $O(|\ptext|^2*log(log(|\ptext|)))$ (see \cite{Bespamyatnikh00enumeratinglongest}).\\

In the base case,
we are looking for a match for the first factor.
Each factor is either ascent or descent.
If the factor is an ascent (descent)
we have to find an increasing (decreasing) subsequence
in the text of same size or longest that
the size of the factor,
and to find the optimal solution
we must assure that the last element
of the sequence is minimal (maximal).\\
For the induction, it is the same idea
except that we must assure that
every element of the subsequence is superior (inferior)
to the maximum (minimal) element of the rest of the match
ie inferior (superior) to the minimal (maximal) element of the rest of the match
which is given by $LM$ of
the previous factor.\\

There exists a match of $\pmotif$ in $\ptext$ if and only if
there exists a $LM(n,i)\neq 0$ for $1 \leq i \leq |\ptext|$,
with $n$ the number of factor in $\pmotif$.
Moreover the basic case can be computed in $O(|\ptext|^2)$
and the induction on $O(|\pmotif||\ptext|^2)$.
\qed
\end{proof}

%%%%%%%%%%%%%%%%%%%%%%%%%%%%%%%%%%%%%%%%%%%%%%%%%%%%%%%%%%%%%%%%%%%%%%%%%%%%%%%

\section{$(213,231)$-avoiding bivincular patterns}
	\label{section:bivincular}

		This section is devoted to presenting a polynomial-time algorithm for deciding whether
	a $(213,231)$-avoiding bivincular pattern occurs in a permutation.

	\begin{proposition}
		\label{Proposition:bivincular pattern}
			Let $\sigma$ be a $(213,231)$-avoiding bivincular pattern of length $k$
			and $\pi \in S_n$.
			One can decide in $O(kn^3)$ time
			and $O(kn^3)$ space if $\pi$ contains $\sigma$.
	\end{proposition}


\begin{proof}
We consider the following problem :\\
Given a bivincular motif $\pbmotif$ with $\pmotif \in \Av{n_\pmotif}{231,213} $, and a text $\ptext \in \Av{n_\ptext}{231,213}$, $i$,$j$ , $i<j$.\\
$$
\PM{\pbmotif}{\ptext}{\min}{\max}{i}{j}=
\begin{cases}
	true 	& \text{If $\pbmotif[i:]$ is a motif of $\ptext[j:]$}\\
			  & \text{with every element in $[\min,\max]$}\\
	false 	& otherwise\\
\end{cases}
$$
$\PM{\pbmotif}{\ptext}{\min}{\max}{i}{j}$ is closed under induction.
It can be solved by means of the following relations:\\

\textbf{BASE :} \\
if $n_\pmotif \notin  \x $ and $n_\pmotif \notin  \y$ : \\
$
\PM{\pbmotif}{\ptext}{\min}{\max}{n_\pmotif}{j} =
\begin{cases}
	true 	& \text{if $\min<\ptext[j]<\max$}\\
	false 	& otherwise\\
\end{cases}
$
%$\PM{\pbmotif}{\ptext}{\min}{\max}{n_\pmotif+1}{j}= true $
\\

if $n_\pmotif \in  \x $ and $n_\pmotif \notin  \y$ : \\
$
\PM{\pbmotif}{\ptext}{\min}{\max}{n_\pmotif}{n_\ptext} =
\begin{cases}
	true 	& \text{if $\min<\ptext[n_\ptext]<\max$}\\
	false 	& otherwise\\
\end{cases}
$
%$\mathbb{P}_{\pbmotif,\min,\max}^{\ptext}(n_\pmotif+1,n_\ptext)= true $
\\

if $n_\pmotif \notin  \x$  and $n_\pmotif \in  \y$ and $\pmotif[n_\pmotif]$ is the maximal element: \\
$
\PM{\pbmotif}{\ptext}{\min}{\max}{n_\pmotif}{j}=
\begin{cases}
	true 	& \text{if $\min<\ptext[j]<\max$ and }\\
			&\text{$\ptext[j]$ is the maximal element}\\
	false 	& otherwise\\
\end{cases}
$
%$\mathbb{P}_{\pbmotif,\min,\max}^{\ptext}(n_\pmotif+1,j)= true $\\

if $n_\pmotif \in  \x$  and $n_\pmotif \in  \y$ and $\pmotif[n_\pmotif]$ is the maximal element: \\
$
\PM{\pbmotif}{\ptext}{\min}{\max}{n_\pmotif}{n_\ptext}=
\begin{cases}
	true 	& \text{if $\min<\ptext[n_\ptext]<\max$and }\\
			&\text{$\ptext[n_\ptext]$ is the maximal element}\\
	false 	& otherwise\\
\end{cases}
$
%$\mathbb{P}_{\pbmotif,\min,\max}^{\ptext}(n_\pmotif+1,n_\ptext)= true $\\

\textbf{STEP :} \\
$
\PM{\pbmotif}{\ptext}{\min}{\max}{i}{j}=
\begin{cases}
\bigcup_{j<k}
			\PM{\pbmotif}{\ptext}{\ptext[j]}{\max}{i+1}{k}
				& \text{if $\stripe{\pmotif}{i}=\ustep$
				and $i \notin \x$}\\
				& \text{and $i \notin \y$ }\\
				& \text{and $\min<\ptext[j]<\max$} \\
			&\\

			\bigcup_{j<k}
			\PM{\pbmotif}{\ptext}{\min}{\ptext[j]}{i+1}{k}
				& \text{if $\stripe{\pmotif}{i}=\dstep$}\\
				& \text{and $i \notin \x$ and $i \notin \y$  }\\
				& \text{and $\min<\ptext[j]<\max$} \\
			&\\

			\PM{\pbmotif}{\ptext}{\ptext[j]}{\max}{i+1}{j+1}
				& \text{if $\stripe{\pmotif}{i}=\ustep$}\\
				& \text{and $i \in \x$ and $i \notin \y$}\\
				& \text{and $\min<\ptext[j]<\max$} \\
			&\\

			\PM{\pbmotif}{\ptext}{\min}{\ptext[j]}{i+1}{j+1}
				& \text{if $\stripe{\pmotif}{i}=\dstep$}\\
				& \text{and $i \in \x$ and $i \notin \y$ }\\
				& \text{and $\min<\ptext[j]<\max$} \\

			&\\

			\PM{\pbmotif}{\ptext}{\ptext[j]}{\max}{i+1}{\ptext^{-1}[\ptext[j]+1]}
				&\text{if $\stripe{\pmotif}{i}=\ustep$}\\
				&\text{and $i \notin \x$ and $i \in \y$}\\
				& \text{and $\min<\ptext[j]<\max$} \\
				&\text{and $\ptext^{-1}[\ptext[j]+1]>j$}\\
			&\\


			\PM{\pbmotif}{\ptext}{\min}{\ptext[j]}{i+1}{\ptext^{-1}[\ptext[j]-1]}
				&\text{if $\stripe{\pmotif}{i}=\dstep$}\\
				&\text{and $i \notin \x$ and $i \in \y$ }\\
				& \text{and $\min<\ptext[j]<\max$} \\
				&\text{and $\ptext^{-1}[\ptext[j]-1]>j$}\\
			&\\

			\PM{\pbmotif}{\ptext}{\ptext[j]}{\max}{i+1}{j+1}
				&\text{if $\stripe{\pmotif}{i}=\ustep$}\\
				&\text{and $i \in \x$ and $i \in \y$}\\
				&\text{and $\ptext[j]+1=\ptext[j+1]$ }\\
				& \text{and $\min<\ptext[j]<\max$} \\
			&\\

			\PM{\pbmotif}{\ptext}{\min}{\ptext[j]}{i+1}{j+1}
				& \text{if $\stripe{\pmotif}{i}=\dstep$}\\
				&\text{and $i \in \x$ and $i \in \y$}\\
				&\text{and $\ptext[j]-1=\ptext[j+1]$ }\\
				& \text{and $\min<\ptext[j]<\max$} \\
			&\\


			False & Otherwise\\
\end{cases}
$

At each step (i,j),
If $i \notin \x$ and $i \notin \y$,
we match the current element of $\pmotif$ ($\pmotif[i]$)
with the current element of $\ptext$ ($\ptext[j]$), if possible.
Then we match $\pmotif[i+1:]$ with
every suffixes $\ptext$ starting after $j$.\\

If $i \in \x$ and $i \notin \y$,
we match $\pmotif[i]$ to $\ptext[j]$,
then we match $\pmotif[i+1:]$ with $\ptext[j+1:]$
with the condition that
$\pmotif[i+1]$ is matched to $\ptext[j+1]$.\\

If $i \notin \x$, $i \in \y$ and i is a descent,
we match $\pmotif[i]$ to $\ptext[j]$,
then we match $\pmotif[i+1:]$
with $\ptext[\ptext^{-1}[\ptext[j]-1]:]$
with the condition that
$\pmotif[i+1]$ is matched to the element $\ptext[j]-1$.\\

If $i \notin \x$, $i \in \y$ and i is an ascent,
we match $\pmotif[i]$ to $\ptext[j]$,
then we match $\pmotif[i+1:]$
with $\ptext[\ptext^{-1}[\ptext[j]+1]:]$
with the condition that
$\pmotif[i+1]$ is matched to the element $\ptext[j]+1$.\\

If $i \in \x$, $i \in \y$ and i is a descent,
we match $\pmotif[i]$ to $\ptext[j]$,
then we match $\pmotif[i+1:]$ with $\ptext[j+1:]$
with the condition that
$\pmotif[i+1]$ is matched to $\ptext[j+1]$,
and that $\ptext[j+1]=\ptext[j]-1$.\\

If $i \in \x$, $i \in \y$ and i is an ascent,
we match $\pmotif[i]$ to $\ptext[j]$,
then we match $\pmotif[i+1:]$ with $\ptext[j+1:]$
with the condition that
$\pmotif[i+1]$ is matched to $\ptext[j+1]$,
and that $\ptext[j+1]=\ptext[j]+1$.\\		
		
There are 6 cases to consider to search for a match. The 6 cases depend on if $0 \in \x$, $0 \in \y$ and if the first element of the stripe of the motif is ascent or descent :
\begin{itemize}
	\item If $0 \in \x$ and $0 \notin \y$ there exists a match if and only if there is a match starting with left most element of the text:\\ $\bigcup_{0<\min<\max<n_\ptext}
	\PM{\pbmotif}{\ptext}{\min}{\max}{0}{0}$ is true.
	
	\item If $0 \notin \x$ and $0 \notin \y$ there exists a match if and only if there exists a match stating at any index: \\	$\bigcup_{0<k<n_\ptext,0<\min<\max<n_\ptext}\PM{\pbmotif}{\ptext}{\min}{\max}{0}{k}$ is true.
	
	\item If $0 \in \x$, $0 \in \y$ and the first element of the stripe of the motif is ascent there exists a match if and only if there is a match starting with the left most element  of $\ptext$, and the left most element  of $\ptext$ is the minimal element  of $\ptext$:\\ $\bigcup_{0<\min<\max<n_\ptext}\PM{\pbmotif}{\ptext}{\min}{\max}{0}{0}$  is true and the left most element of $\ptext$ is the minimal element of $\ptext$.
	
	\item If $0 \notin \x$, $0 \in \y$ and the first element of the stripe of the motif is ascent there exists a match if and only if there exists a match stating at the index of the minimal element of $\ptext$ : if $j_{\min}$ is the index of the minimal element then \\	$\bigcup_{0<\min<\max<n_\ptext}\PM{\pbmotif}{\ptext}{\min}{\max}{0}{j_{\min}}$ is true.	
			
	\item If $0 \in \x$, $0 \notin \y$  and the first element of the stripe of the motif is descent then there exists a match if and only if there is a match starting with the left most element  of $\ptext$ and the left most element  of $\ptext$ is the maximal element  of $\ptext$:\\ $\bigcup_{0<\min<\max<n_\ptext}\PM{\pbmotif}{\ptext}{\min}{\max}{0}{0}$ is true and the left most element of $\ptext$ is the maximal element of $\ptext$.
	
	\item If $0 \notin \x$, $0 \notin \y$ and the first element of the stripe of the motif is descent there exists a match if and only if there exists a match stating at the index of the maximal element of $\ptext$ : if $j_{\max}$ is the index of the maximal element then  \\	$\bigcup_{0<\min<\max<n_\ptext}\PM{\pbmotif}{\ptext}{\min}{\max}{0}{j_{\max}}$ is true.	

\end{itemize}



\end{proof}


%%%%%%%%%%%%%%%%%%%%%%%%%%%%%%%%%%%%%%%%%%%%%%%%%%%%%%%%%%%%%%%%%%%%%%%%%%%%%%%

\section{Longest Subsequence Avoiding \\(231,213) for a Permutation}
\label{section:LCS}

	In this section we present an algorithm to solve the problem
	of the longest subsequence avoiding (231,213).
	To do so, we need the set of element
	that are matched to an ascent and the
	set of element that are matched to a descent.
	For a permutation $\pi$ of size $n$, we define
	$P(\pi) = \{i | \stripe{\pi}{i} = \ustep \} \cup \{n\}$ and
	$M(\pi) = \{i | \stripe{\pi}{i} = \dstep \} \cup \{n\}$.\\

	\begin{proposition}
	\label{proposition:longestIncreasingSubsequence}
	If $s$ is a longest subsequence avoiding (213,231) with last element at index $f$ in $\pi$ then
	$P(\pi)$ is a longest increasing subsequence with last element at index $f$ and
	$M(\pi)$ is a longest decreasing subsequence with last element at index $f$.
	\end{proposition}

	\begin{proof}
	Let's $s$ be a longest subsequence avoiding (213,231) with last element at index $f$ in $\pi$,
	suppose that $P(\pi)$ is not a longest increasing subsequence with last element at index $f$. Let's $s_m$ be a longest increasing subsequence with last element $f$.
	Thus $|s_m|>|P(\pi)|$, clearly the sequence $s_m \cup M(\pi)$
	is avoiding (213,231) and is longer than $s$ witch is not possible.
	The same idea goes for the demonstration that $M(\pi)$ is the longest decreasing subsequence.
	\end{proof}

	\begin{proposition}
	Given a permutation $\pi$,
	finding the longest subsequence avoiding (231,213)
	can be done in $O(|\ptext|log(log(|\ptext|)))$ time and in $O(n)$ space.
	\end{proposition}

	\begin{proof}
	The proposition \ref{proposition:longestIncreasingSubsequence} lead to algorithm
	where we have to compute longest increasing and decreasing subsequence ending at every index. Then finding the maximum sum of longest increasing and decreasing subsequence ending at the same index.
	Computing the longest increasing and decreasing can be done in $O(|\ptext|*log(log(|\ptext|)))$ time and $O(n)$ space (see \cite{Bespamyatnikh00enumeratinglongest}), then finding the maximum can be done in linear time.
	\end{proof}


\section{Longest Subsequence Avoiding \\(231,213) Common for Two Permutation}

In this section we present an algorithm to find
the longest common subsequence avoiding (231,213)
between two permutations.\\

\begin{proposition}
Given two permutation $\pi_1$ and $\pi_2$,
the longest common subsequence
avoiding (231,213) can be compute in $O(|\pi_1|^3|\pi_2|^3)$ time and space.
\end{proposition}

\begin{proof}
Consider the following problem,
that compute the longest stripe common to $\pi_1$ and $\pi_2$.
Given two permutation $\pi_1$ and $\pi_2$.\\

$\LCS{\pi_1}{\min_1}{\max_1}{\pi_2}{\min_2}{\max_2}{i_1}{i_2}$
= max$\{$ $|s|$ $|$
s is a pattern occurring
in  $\pi_1[i_1:]$
by the subsequence $s_1$ and $\min(s_1)=\min_1$ and $\max(s_1)=\max_1$
and s is occurring
in  $\pi_2[i_2:]$
by the subsequence $s_2$ and $\min(s_2)=\min_2$ and $\max(s_2)=\max_2$
$\}$\\

We show that this family of problems are closed under induction.\\


\textbf{BASE :} \\
$
\LCS{\pi_1}{\min_1}{\max_1}{\pi_2}{\min_2}{\max_2}{|\pi_1|}{|\pi_2|} =
\begin{cases}
	1 & \text{if $\min_1=\pi_1[j]=\max_1$
	}\\
	& \text{ and $\min_2=\pi_2[j]=\max_2$}\\
	0 & otherwise\\
\end{cases}
$


\textbf{STEP :} \\

$
\LCS{\pi_1}{\min_1}{\max_1}{\pi_2}{\min_2}{\max_2}{i_1}{i_2}=max
\begin{cases}
	\LCS{\pi_1}{\min_1}{\max_1}{\pi_2}{\min_2}{\max_2}{i_1}{i_2+1} \\
	\\
	\LCS{\pi_1}{\min_1}{\max_1}{\pi_2}{\min_2}{\max_2}{i_1+1}{i_2} \\
	\\
	\match{\pi_1}{\min_1}{\max_1}{\pi_2}{\min_2}{\max_2}{i_1}{i_2}
\end{cases}
$

with \\
$
\match{\pi_1}{\min_1}{\max_1}{\pi_2}{\min_2}{\max_2}{i_1}{i_2}=
\begin{cases}
1+\LCS{\pi_1}{\pi_1[i_1]}{\max_1}{\pi_2}{\pi_2[i_2]}{\max_2}{i_1}{i_2+1}
	& \text{$\pi_1[i_1]<\min_1$ } \\
	& \text{and $\pi_2[i_2]<\min_2$} \\

&\\

1+\LCS{\pi_1}{\min_1}{\pi_1[i_1]}{\pi_2}{\min_2}{\pi_2[i_2]}{i_1+1}{i_2}
	& \text{$\pi_1[i_1]>\max_1$ } \\
	&\text{and $\pi_2[i_2]>\max_2$}\\

&\\

0 	& \text{otherwise}\\
\end{cases}
$


For every pair $i,j$ we either ignore the element of $\pi_1$,
either ignore the element of $\pi_2$,
either we match as the same step (if possible).
Those relation lead to a $O(|\pi_1|^3|\pi_2|^3)$ time and $O(|\pi_1|^3|\pi_2|^3)$ space algorithm.
\end{proof}



%	\section{Related problems}
%
%		\textbf{Proposition 3.1} Longest Minimal Common Supersequence, Shortest Maximal Common Subsequence, and Shortest Maximal Common Non-Supersequence are MAX SNP-hard.\\
%
%		Clearly every problem over (231,213)-Avoiding permutation
%		are equivalent to problem over binary word.
%		The proposition follow what we know about those problem over binary alphabet
%		\cite{Middendorf1995317}.\\



	\section{Conclusion}




\bibliography{bibli}{}
\bibliographystyle{plain}


\end{document}
