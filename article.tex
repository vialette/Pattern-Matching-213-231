\documentclass[12pt, a4paper]{article}

\usepackage{amsfonts}
\usepackage{amsmath}

\usepackage[all]{xy}

\usepackage{algorithm}
\usepackage[noend]{algpseudocode}

% 231,213
\newcommand{\myPattern}{231,231}

% avoid set
\DeclareMathOperator{\Avd}{Av}
\newcommand\Av[2]{\Avd_{{#1}}({#2})}

% le set de toutes les permitaton
\newcommand{\Perm}[1]{\mathcal{S}_{#1}}

% permutation text
\newcommand{\ptext}{\pi}

% permutation motif
\newcommand{\pmotif}{\sigma}


\newcommand{\x}{X}
\newcommand{\y}{Y}
\newcommand{\bmotif}{(\sigma,\x,\y)}

% compteur pour les proposition, definition etc...
\newcounter{num}
\setcounter{num}{0}
\newcommand{\num}{\stepcounter{num} }
\newcommand{\numl}[1]{\refstepcounter{num}\label{#1}}

%downstep
\newcommand{\dstep}{d}

%upstep
\newcommand{\ustep}{u}

\title{Pattern matching for (231,213) avoiding permutations}
\author{Both Emerite NEOU, Romeo RIZZI}
\date{}



\begin{document}
	\maketitle 
	\tableofcontents
	
	\abstract{
	  	Given two permutations, 
	  	a text $\ptext$ 
	  	and a motif $\pmotif$,
	  	the \textbf{permutation pattern problem}
	  	asks if there exists a normalized subsequence of $\ptext$
	  	which equals $\pmotif$.
	  	We study this problem for the permutation 
	  	avoiding (231,231).
	} 

	\section{Introduction}

	
%	The permutation pattern problem asks if there exists a normalized subsequence of a text $\ptext$ which equals to a motif $\pmotif$.
%	If so we write $\pmotif \preceq \ptext$.
%	If not we say that $\ptext$ \textbf{avoid} $\pmotif$.
%	This was first introduced by Knuth \cite{Knuth:1997:ACP:260999}
%	who proves that permutation sortable by stacks
%	avoid the motif $231$. 
%	The problem is NP-hard in general,
%	but for some classes of permutation,
%	polynomial time solution exists.
%	For instance the separable permutation \cite{Bose93patternmatching}.
%	We contribute studying this problem 
%	and related one for the class of permutation avoiding (231,213). 
%	This is interesting because this class is in linear bijection between the classes of (123,132),(132,213) and (132,213)
%	avoiding permutation, and all the classes obtainable by reversal or/and
%	complementation (see \cite{Simion1985383}).\\
%	 
%
%	
%	This paper is organized as follows.
%	Section 2 presents the needed definition.
%	Section 3 solves the permutation pattern problem with
%	(213,231) avoiding permutations.
%	Section 4 focus on the problem the
%	related problem of computing a longest permutation avoiding (213,231).
%	Finally, section 5 summarizes 
%	results for related problems. 
	

	
	

	
	\section{Definition}

		This section presents the decomposition used and some notation.\\	
		
		We denote by $\Av{n}{213,231}$ the class
		of permutation of size $n$ which avoid 
		the motif $213$ and $231$. 
		$\Perm{n}$ the permutation of size $n$.
		$[n]$ the set $\{1,2,...,n\}.$
		For a permutation p, let $p[i:j]$, 
		be the permutation from element $i$ to $j$,
		$p[:j]$ be the prefix ending at index $j$
		and $p[i:]$ be the suffix starting at index $i$.\\
		
		Given two permutations
		$\pmotif$ and $\ptext$,
		we say that $\pmotif$
		is a motif/match of
		$\ptext$,
		if and only if
		there exists 
		and increasing function 
		$\sigma:[|\pmotif|] \rightarrow [|\pmotif|]$
		such that : $\forall i$ and 
		$\forall j$ $i,j \in |\pmotif| $
		if $\pmotif(i)<\pmotif(j)$
		then $\ptext(\sigma(i))<\ptext(\sigma(j))$.\\
		
		%\subsection{Stripe}
		
		A \textbf{Stripe} of a permutation
		 is a mapping into \textbf{up step} (noted $\ustep$) 
		or a \textbf{down step} (noted $\dstep$).
		An element $i$ of a permutation $p$ is an up step if and only if $i$ is not the last element and $p(i)<p(i+1)$.
		An element $i$ of a permutation $p$ is a down step if and only if $i$ is not the last element and $p(i)>p(i+1)$.	
		By convention the last element is mapped as the second to last element.\\ 	
		Formally a Stripe is a function $\mathcal{S} : [n] \to \{\dstep,\ustep\}$ define as :\\
		$ \mathcal{S}(p,i) = \left\{ 
		  \begin{array}{l l}
		    \ustep & \quad \text{if $1 \leq i<n$ and $p(i+1)-p(i)>0$ }\\
		    \dstep & \quad \text{if $1 \leq i<n$ and $p(i+1)-p(i)<0$ }\\
		     \mathcal{S}(p,n-1) & \quad \text{if $i=n$}\\
		  \end{array} \right. $\\
		  
		By extension we define  $\mathcal{S} : \Perm{n} \to \{\dstep,\ustep\}^{n}$ such as :
		$\mathcal{S}(p)=\mathcal{S}(p,1)\mathcal{S}(p,2)...\mathcal{S}(p,n)$.
		We call $\mathcal{S}(p)$ the stripe of $p$.
		For example the stripe of the permutation $(12,11,10,1,2,3,4,9,8,5,6,7)$, 
		is $\dstep,\dstep,\dstep,\ustep,\ustep,\ustep,\ustep,\dstep,\dstep,\ustep,\ustep,\ustep$.\\
		  
%		\textbf{Corollary \num} The stripe function
%		for $\C{n}$ are in bijection with with $\{\ustep,\dstep\}^{n-1}$.\\
	  			 		  	
%		\textbf{Proposition \numl{updown}} Let $p \in \C{n}$,
%		$\forall x,y \in [n]$, 
%		$\mathcal{S}(p,x)=\ustep$
%		and $\\mathcal{S}(p,y)=\dstep$
%		then $p(x)<p(y)$.\\
		

		
		Formally we say that the stripe of $\pmotif \in \Perm{m}$ is a motif
		of a stripe of $\ptext \in \Perm{n}$ if and only if there exist an increasing function $\phi : [|\pmotif|] \rightarrow [|\pmotif|]$
		such that : \\
		$\forall i \in [|\pmotif|]$, $\mathcal{S}(\pmotif,i) = \mathcal{S}(\ptext,\phi(i))$\\
		If a stripe $R_2$ is a motif of a stripe $R_1$,
		it means that there exists
		a subsequence of $R_1$ such as
		this sequence is equal to $R_2$.
%		We say that a set is a match 
%		if and only if there exists a function such as define above
%		and the set is the set of image of this function.\\
		For example, the stripe $(\dstep_1,\dstep_2,\dstep_3,\dstep_4,\ustep_1,\ustep_2)$ is a motif of  the stripe \\
		$(\dstep_1,\dstep_2,\dstep_3,\ustep_1,\ustep_2,\ustep_3,\ustep_4,\dstep_4,\dstep_5,\ustep_5,\ustep_6,\ustep_7)$ ,
		because we can match \\
		$(\dstep_1,\dstep_2,\dstep_3,\dstep_4,(\ustep_1,\ustep_2)$ to $(\dstep_1,\dstep_2,\dstep_3,\dstep_4,\ustep_5,\ustep_6)$.
		\begin{figure}[h!]
			\begin{displaymath}
				\xymatrix@=0.3cm{
					\dstep_1 \ar[dd] & \dstep_2 \ar[dd] & \dstep_3 \ar[dd] & \dstep_4\ar[ddrrrr]  & \ustep_1 \ar[ddrrrrr]& \ustep_2 \ar[ddrrrrr]  \\
					\\
					\dstep_1 & \dstep_2 & \dstep_3 & \ustep_1 & \ustep_2 & \ustep_3 & \ustep_4 & \dstep_4 & \dstep_5 & \ustep_5 & \ustep_6 & \ustep_7 & \\
				}
			\end{displaymath}     
		\end{figure}		
		But the stripe $(\dstep,\ustep,\dstep,\dstep,\dstep)$ is not a motif of the stripe\\ $(\dstep_1,\dstep_2,\dstep_3,\ustep_1,\ustep_2,\ustep_3,\ustep_4,\dstep_4,\dstep_5,\ustep_5,\ustep_6,\ustep_7)$.\\
		
		A \textbf{bivincular motif} generalize
		a motif. 
		A bivincular motif allows 
		to force element to be adjacent in position
		or/and in value.
		A bivincular motif
		is a motif with two set $\bmotif$,
		the first set contains the first index 
		of two consecutive indexes
		and the second set contains the first indexes
		of the two consecutive elements.
		Formally, Let $\bmotif$ 
		be a bivincular motif 
		and $\ptext$ a permutation.
		if $i \in \x$,
		and $i$ is matched with $j$ in $\ptext$
		then $i+1$ must be matched with $j+1$,
		with the special case of $0$ and $|\pmotif|$ :
		if $0 \in \x$ then the first element of $\pmotif$
		must be matched with the first element of $\ptext$,
		and if $|\pmotif| \in \x$ then the last element of $\pmotif$
		must be matched with the last element of $\ptext$.
		if $i \in \y$,
		and $i$ is matched with $j$ in $\ptext$
		then $i+1$ must be matched with $l$
		such as $\ptext(l)=\ptext(j)+1$,
		with the special case of $0$ and $|\pmotif|$ :
		if $0 \in \y$ then the minimal element of $\pmotif$
		must be matched with the minimal element of $\ptext$,
		and if $|\pmotif| \in \y$ then the maximal element of $\pmotif$
		must be matched with the maximal element of $\ptext$.		
			
	\section{Pattern Matching Problem With Text and Motif Avoiding (231,213)}

		In this section we focus 
		on the problem of pattern matching if
		both the motif and the text
		avoid (231,213).
		We first exhibit some properties of a stripe 
		of a permutation avoiding (231,213).\\

		\textbf{Lemma  \numl{firstelement} \thenum} If a permutation is in $\Av{n}{}$ 
		then the first element is either
		the minimal or the maximal element.\\
		
		\textbf{Proof} Suppose that we have
		$p \in \mathcal{S}_{n}$, and
		$p(1) \neq 1 $ and $p(1) \neq n$ then we have either
		$p(1) ... n ... 1$ or $p(1) ... 1 ... n $, either the motif 231 or 213, which is not possible by hypothesis. $\Box$\\
		
		\textbf{Corollary  \num \thenum} A permutation $p \in \Perm{n}$ is in $\Av{n}{213,231}$ if and only if for $1 \leq i \leq n$, 
		$p[i:]$ start either with the maximal element
		or the minimal element of $p[i:]$.\\
	
		\textbf{Corollary \numl{stripe} \thenum}  If $p \in \Av{n}{213,231}$ its stripe can be define such as, \\
		$\mathcal{S}(p,i) = \left\{ 
  		\begin{array}{l l}
   			\ustep & \quad \text{if $p(i)$ is the minimal element of $p[i:]$}\\    
   			\dstep & \quad \text{if $p(i)$ is the maximal element of $p[i:]$}\\
  		 	\mathcal{S}(p,n-1) & \quad \text{if $i=n$}\\
	 	\end{array} \right. 
	 	$\\
	 
		\textbf{Lemma \numl{updown}  \thenum} Let $p \in \Av{n}{213,231}$,
		$\forall i,j \in [n]$, 
		$i<j$, 
		if $\mathcal{S}(p,i)=\ustep$
		and $\mathcal{S}(p,j)=\dstep$
		then $p(i)<p(j)$.\\ 
		
		In other word, for every index $i$ and $j$ of an (231,213) avoiding permutation p,
		if $i$ is a up step and
		if $j$ is a down step
		then $p(i)<p(j)$.\\
		
		\textbf{Proof} By induction on the size of the permutation :
		the proposition is true for the permutations $(1,2)$ and $(2,1)$.
		Suppose the proposition true for $\Av{n}{213,231}$.
		Let $p \in \Av{n}{213,231}$, 
		we show that the proposition holds for $p$.
		Let $p'$ be the permutation $p[2:]$.
		$p' \in \Av{n}{213,231}$, so
		every element of $p'$ verify our lemma (by induction).
		We still need to compare the first element with the rest.
		But the first element
		is the maximal (minimal) element of all the permutation 
		(by Lemma \ref{firstelement}).
		Moreover if the first element is the maximal (minimal)
		it will mapped to an down step (up step).
		So the proposition is also true
		for the first element.
		$\Box$\\
		
		\textbf{Lemma \numl{upup} \thenum} Let $p \in \Av{n}{213,231}$,
		$\forall i,j \in [n]$, 
		$i<j$, 
		if $\mathcal{S}(p,i)=\ustep$
		and $\mathcal{S}(p,j)=\ustep$
		then $p(i)<p(j)$.\\ 
		
		In other word, for every index $i$ and $j$ of an (231,213) avoiding permutation p,
		if $i$ is a up step 
		and $j$ is a up step
		and $ i<j$
		then $p(i)<p(j)$.\\
		
		%\textbf{Proof} Same as before.\\

%		\textbf{Proof} By induction on the size of the permutation :
%		the proposition is true for the permutations $(1,2)$ and $(2,1)$.
%		Suppose the proposition true for $\C{n}$.
%		Let $p \in \C{n+1}$, 
%		we show that the proposition holds for $p$.
%		Let $p'$ be the permutation $p[2:]$.
%		$p' \in \C{n}$, so
%		every element of $p'$ verify our lemma (by induction).
%		We still need to compare the first element with the rest.
%		Except that the first element
%		is the maximal (minimal) element of all the permutation 
%		(by Corollary \ref{stripe}).
%		Moreover if the first element is the maximal (minimal)
%		it will mapped to an down step (up step).
%		So the proposition is also true
%		for the first element.
%		$\Box$\\

		
		\textbf{Lemma \numl{downdown} \thenum}  Let $p \in \Av{n}{213,231}$,
		$\forall i,j \in [n]$,
		$i<j$,
		if $\mathcal{S}(p,i)=\dstep$
		and $\mathcal{S}(p,j)=\dstep$
		then $p(i)>p(j)$.\\ 

		In other word, for every index $i$ and $j$ of an (231,213) avoiding permutation p,
		if $i$ is a down step
		and $j$ is a down step
		and $ i<j$
		then $p(i)>p(j)$.\\

		%\textbf{Proof} Same as before.\\
		
%		\textbf{Proof} By induction on the size of the permutation :
%		the proposition is true for the permutations $(1,2)$ and $(2,1)$.
%		Suppose the proposition true for $\C{n}$.
%		Let $p \in \C{n+1}$, 
%		we show that the proposition holds for $p$.
%		Let $p'$ be the permutation $p[2:]$.
%		$p' \in \C{n}$, so
%		every element of $p'$ verify our lemma (by induction).
%		We still need to compare the first element with the rest.
%		Except that the first element
%		is the maximal (minimal) element of all the permutation 
%		(by Corollary \ref{stripe}).
%		Moreover if the first element is the maximal (minimal)
%		it will mapped to an down step (up step).
%		So the proposition is also true
%		for the first element.
%		$\Box$\\

		\textbf{Proposition \numl{patternMatchingStripe} \thenum} Let $\ptext, \pmotif \in \Av{n}{213,231}$.
		$\pmotif$ is a motif of $\ptext$ if and only if the stripe of $\pmotif$ is a motif of the stripe of $\ptext$.\\
		
		\textbf{Proof} \\
		$\Rightarrow$ Suppose that $\pmotif$ is a motif of $\ptext$
		by subsequence $s$.
		By definition $s$ normalized is equal to
		$\pmotif$ so they have the same stripe.\\ 
		$\Leftarrow$ 
		Let $\ptext \in \Perm{n}$, 
		$R_1$ be its stripe, 
		$\pmotif \in \Perm{m}$
		and $R_2$ its stripe.
		Let $\phi$ be the function
		that match $R_2$ to $R_1$.
		We define $\varphi$ as, 
		$\forall x \in [m]$,
		$\varphi(x) = \phi(x)$.
		Suppose that $\exists x,y \in [|\pmotif|]$, 
		$\pmotif(x)<\pmotif(y)$ 
		and $\ptext(\varphi(x))>\ptext(\varphi(y))$,
		without lost of generality we can suppose that $x<y$,
		Let see all the case possible that can make this 
		hypothesis possible :
		\begin{itemize}
		
			\item if $\pmotif(x)<\pmotif(y)$, 
			$\mathcal{S}(\pmotif,x)=\ustep$ 
			and $\mathcal{S}(\pmotif,y)=\ustep$
			(if x and y are increasing in $\pmotif$),
			then $\mathcal{S}(\ptext,\varphi(x))=\ustep$ and $\mathcal{S}(\ptext,\varphi(y))=\ustep$,
			(the match are also increasing).
			As $x<y$ and as $\varphi$ is an increasing function, 
			$\varphi(x) < \varphi(y)$.
			And thus, by \textbf{Lemma \ref{upup}},
			$\ptext(\varphi(x)) < \ptext(\varphi(y))$ .
			Which contradict our hypothesis,
			so this case is not possible.
		
			\item if $\pmotif(x)<\pmotif(y)$, 
			$\mathcal{S}(\pmotif,x)=\ustep$ 
			and $\mathcal{S}(\pmotif,y)=\dstep$
			then $\mathcal{S}(\ptext,\varphi(x))=\ustep$ and $\mathcal{S}(\ptext,\varphi(y))=\dstep$,
			As $x<y$ and as $\varphi$ is an increasing function, 
			$\varphi(x) < \varphi(y)$.
			And thus, by \textbf{Lemma \ref{updown}},
			$\ptext(\varphi(x)) < \ptext(\varphi(y))$ .
			Which contradict our hypothesis,
			so this case is not possible.			

			\item if $\pmotif(x)>\pmotif(y)$, 
			$\mathcal{S}(\pmotif,x)=\dstep$ 
			and $\mathcal{S}(\pmotif,y)=\dstep$
			then $\mathcal{S}(\ptext,\varphi(x))=\dstep$ and $\mathcal{S}(\ptext,\varphi(y))=\dstep$,
			As $x<y$ and as $\varphi$ is an increasing function, 
			$\varphi(y) < \varphi(x)$.
			And thus, by \textbf{Lemma \ref{downdown}},
			$\ptext(\varphi(y)) > \ptext(\varphi(x))$ .
			Which contradict our hypothesis,
			so this case is not possible.
			
			\item if $\pmotif(x)>\pmotif(y)$, 
			$\mathcal{S}(\pmotif,x)=\dstep$ 
			and $\mathcal{S}(\pmotif,y)=\ustep$
			then $\mathcal{S}(\ptext,\varphi(x))=\dstep$ and $\mathcal{S}(\ptext,\varphi(y))=\ustep$,
			As $x<y$ and as $\varphi$ is an increasing function, 
			$\varphi(y) < \varphi(x)$.
			And thus, by \textbf{Lemma \ref{updown}},
			$\ptext(\varphi(y)) > \ptext(\varphi(x))$ .
			Which contradict our hypothesis,
			so this case is not possible.$\Box$
			
			
		\end{itemize}
		So our hypothesis is not realisable.So our assertion is true. $\Box$\\
		
		\textbf{Proposition  \num \thenum} Let $\pmotif$ and $\ptext$
		be two permutations avoiding (231,213), 
		we can decide in linear time if $\pmotif$ 
		appear in $\ptext$.\\	
		
		\textbf{Proof}	
		To solve the permutation pattern for a permutation in $\Av{n}{213,231}$, 
		we find a find a match of the stripe of the pattern in the stripe of the text.
		This can be done linearly by a greedy algorithm, 
		by matching a step whenever it is possible.
		And thank to the definition of the definition of a stripe in \textbf{Corollary \ref{stripe}},
		we do not have to compute the stripe advance.
		Therefore we have an linear (online) algorithm.$\Box$\\
				

		

			
%		\subsection{With Motif Avoiding (231,213)}	
%		
%			\textbf{Proposition \num} Let $\pmotif$, $\ptext$ 
%			be two permutation,
%			with  $\pmotif$ a permutation avoiding (231,213), 
%			we can decide in $O(|\pmotif|*|\ptext|^3)$ time
%			and $O(|\ptext|^3) space$
%			if $\pmotif$ 
%			appear in $\ptext$.\\	
%
%			\textbf{Proof}
%			We consider the following problem :\\
%			
%			Given a motif $\pmotif$ avoiding $(231,213)$,
%			a text $\ptext$, $i$,$j$ , $i<j$.\\
%			
%			$\mathbb{P}_{\pmotif,\ptext,min,max}(i,j)= \left\{ 
%				\begin{array}{l l}
%					true & \text{If the sufixe of $\pmotif$ starting at $i$}\\
%						& \text{is a pattern of the suffix of $\ptext$} \\
%						& \text{with every element in $[min,max]$}\\
%		
%					false & otherwise\\
%				\end{array} \right.$\\
%		
%			$\mathbb{P}_{\pmotif,\ptext,min,max}$ is closed under induction. 
%			it can be
%			solved by means of the following relations:\\
%			
%			
%			\textbf{BASE :} \\
%			$\mathbb{P}_{\pmotif,\ptext,min,max}(|\pmotif|,j)= \left\{ 
%					\begin{array}{l l}
%						true & \text{if $min < \ptext(j) < max$
%						}\\
%						false & otherwise\\
%					\end{array} \right. $\\	
%		
%			\textbf{STEP :} \\
%			$\mathbb{P}_{\pmotif,\ptext,min,max}(i,j)= \left\{ 
%					\begin{array}{l l}
%						False & \text{if $\ptext(j)<min$ or $max<\ptext(j)$}\\
%						\bigcup_{j<k}\mathbb{P}_{\pmotif,\ptext,t(j),max}(i+1,k) 
%							& \text{if $\\mathcal{S}(\pmotif,i)=\ustep$
%							}\\
%						\bigcup_{j<k}\mathbb{P}_{\pmotif,\ptext,min,t(j)}(i+1,k) 
%							& \text{if $\\mathcal{S}(\pmotif,i)=\dstep$
%							}\\
%					\end{array} \right. $\\
%			At each step, 
%			we are match the current element of $\pmotif$ ($\pmotif(i)$)
%			with the current element of $\ptext$ ($\ptext(j)$), if possible.
%			then we are match the rest of $\pmotif$ with 
%			every suffixes $\ptext$ starting after $j$.
%			Clearly there is a match if and only if 	$\bigcup_{0<k<|\ptext|,0<min<max<|\ptext|}\mathbb{P}_{\pmotif,\ptext,min,t(j)}(0,k)$
%			is true.   
%			This implies a $O(|\pmotif|*|\ptext|^3)$ time
%			and $O(|\ptext|^3) space$ algorithm.\\ 

		
		\section{Pattern Matching With Motif Avoiding (231,213)}	

			\textbf{Proposition  \num \thenum} Let $\pmotif$, $\ptext$ 
			be two permutation,
			with  $\pmotif$ a permutation avoiding (231,213), 
			we can decide in $O(|\pmotif|*|\ptext|^2)$ time
			and $O(|\pmotif|*|\ptext|^2)$ space
			if $\pmotif$ 
			is a motif of $\ptext$.\\
			
%			\textbf{Proof}
%			The main point is that
%			if we match a up step (down step) element $i_{\pmotif}$ of the motif
%			with an element $i_{\ptext}$ of the text,
%			the optimal way of matching
%			the first previous up step (down step)
%			that appear before an down step (up step)
%			is to choose the
%			maximal (minimal) element possible.
%			Indeed if we do not take
%			the maximal (minimal) element,
%			we may prevent some element
%			in the future matching.\\
			
			For the algorithm we introduce a decomposition,
			we decompose the motif into
			adjacent element of same step in the stripe 
			(called such sequence segment),
			and label them form right to left
			with increasing number.
			Let's call this decomposition $S$.
			For each segment, we call important 
			the leftmost element denoted by $IMP$.
			For example, given the motif $(12,11,10,1,2,3,4,9,8,5,6,7)$,
			$S(1)=(5,6,7)$, $IMP(1)=5$, 
			$S(2)=(9,8)$, $IMP(2)=9$, 
			$S(3)=(1,2,3,4)$, $IMP(3)=1$, 
			$S(4)=(12,11,10)$ and $IMP(4)=12$.\\
			
			Given a up step (down step) segment $S(s)$,
			if we want to
			match the suffix $s_ns_{s-1}...s_1$
			starting with an up step (down step),
			and given that we know every
			match of $s_{s-1}...s_1$,
			the optimal way of matching
			$s_{s-1}...s_1$
			is to choose the match
			that minimize (maximize)
			the maximal (minimal)
			element of the match
			and that allow a match
			of $s_n$.			
			In other word if we want to match 
			$\pmotif[IMP(s):]$
			with $\ptext[i:]$
			and match $IMP(s)$
			with $i$.
			The optimal way
			of matching
			$\pmotif[IMP(s-1):]$
			is to choose the
			minimal (maximal) element $\ptext(i')$
			such that :
			(1) $\pmotif[IMP(s-1):]$ is matched with $\ptext[i':]$
			and $IMP(s-1)$ is matched with $e$.
			(2) The segment $S(s)$ is matched with $\ptext[i:i'-1]$
			and $IMP(s)$ is matched with $i$.
			(3) Every elements of the match of (2),
			is inferior (superior) to the minimal (maximal) element of the match of (1).
			Indeed when we will 
			match $s_{n+1}s_ns_{s-1}...s_1$
			more element will be available,
			because every matched 
			element of $s_{n+1}$
			must be superior (inferior)
			to every matched element
			of $s_{s-1}...s_1$.\\
			  
			
			Consider the following problem :\\
			$LM(s,i)$ = the optimal match of 	
				$IMP(s-1)$ 
				if there exists a match of $\pmotif[IMP(s):]$ 
				with $\ptext[i:]$
				and $IMP(s)$ 
				is matched with element $i$.\\
				
			This problem can be solve by induction.\\
			
			\textbf{BASE :} \\
			$LM(1,i)= \left\{ 
					\begin{array}{l l}
					
						MIN_{i<i'}\{0\} \cup \{i' |\text{ $i'$ such that } & 
						\text{if $\mathcal{S}(\pmotif,IMP(1))=\ustep$}
						\\
						\text{ $LIS(i,i',\ptext(i')) \leq S(1)$}\}&\\			


						&\\
						
						MAX_{i<i'}\{0\} \cup \{i' |\text{ $i'$ such that } &
						\text{if $\mathcal{S}(\pmotif,IMP(1))=\dstep$} \\
						\text{ $LDS(i,i',\ptext(i')) \leq S(1)$}\}&\\			

					\end{array} \right. $\\	
		
			\textbf{STEP :} \\
			$LM(s,i)= \left\{ 
					\begin{array}{l l}
					
						MIN \{0\} \cup  I(s,i) & 
						\text{if $\mathcal{S}(\pmotif,IMP(s))=\ustep$}\\

						
						MAX \{0\} \cup  D(s,i) & 
						\text{if $\mathcal{S}(\pmotif,IMP(s))=\ustep$}\\
	
						
											
					\end{array} \right. $\\
					
			With $I(s,i)$ ($D(s,i)$) the set of element such that
			if  $e \in I(s,i)$ then there exists 
			a match of $\pmotif[IMP(s-1):]$ in $\ptext[e:]$ and
			there is a match of $s$ in $\ptext[i:e-1]$
			with every element inferior (superior) to the match of  
			$\pmotif[IMP(s-1):]$ in $\ptext[e:]$.
			Formally we define $I(s,i)$ and $D(s,i)$ such as:\\
			
			$I(s,i)=\{e|\text{$i<e$ and $LM(s-1,e)\neq 0$}$ \\
			\indent \indent \indent and  $LIS(i,e-1,IMP(s-1)) \leq S(s) \}$ \\
			
			$D(s,i)=\{e|\text{$i<e$ and $LM(s-1,e)\neq 0$}$ \\
			\indent \indent \indent and  $LDS(i,e-1,IMP(s-1)) \leq S(s) \}$ \\
					                       
			With LIS(i,j,k) (LDS(i,j,k)) is the longest increasing (decreasing) sequence in $\ptext$ starting at $i$ and ending at $j$,
			with every element of this sequence
			inferior (superior) to $k$. 
			LIS and LDS can be computed in $O(|\ptext|^2*log(log(|\ptext|)))$ (see \cite{Bespamyatnikh00enumeratinglongest}).\\
			
			In the base case, 
			we are looking for a match for the first segment.
			Each segment is either up step or down step.
			If the segment is an up step (down step)
			we have to find an increasing (decreasing) sequence
			in the text of same size of longest that 
			the size of the segment,
			and to find the optimal solution 
			we must assure that the last element
			of the sequence is minimal (maximal).\\
			For the induction, it is the same idea
			except that we must assure that 
			every element of the sequence is superior 
			to the maximum element of the rest of the match
			or inferior to the minimal element of the rest of the match
			which is given by $LM$ of
			the previous segment.\\
			
			There exists a match of $\pmotif$ in $\ptext$ if and only if
			there exists a $LM(n,i)$ for $1 \leq i \leq |\ptext|$,
			with $n$ the number of segment in $\pmotif$.
			Moreover the basic case can be computed in $O(|\ptext|^2)$
			and the induction on $O(|\pmotif|*|\ptext|^2)$ $\Box$.
			
					
		\section{Pattern Matching With Bivincular Motif Avoiding (231,213)}	
		
			\textbf{Proposition \num \thenum } Let $\bmotif$ be a bivincular motif 
			avoiding (231,213), 
			$\ptext$ a permutation,
			we can decide in $O(|\pmotif|*|\ptext|^3)$ time
			and $O(|\ptext|^3)$ space
			if $\pmotif$ 
			appear in $\ptext$.\\	
			
			\textbf{Proof }
			We consider the following problem :\\
			
			Given a bivincular motif $\bmotif$ avoiding $(231,213)$,
			a text $\ptext$, $i$,$j$ , $i<j$.\\
			
			$\mathbb{P}_{\bmotif,\ptext,min,max}(i,j)= \left\{ 
				\begin{array}{l l}
					true & \text{If $\pmotif[i:]$ is a motif of $\ptext[j:]$}\\
						& \text{with every element in}\\ 
						&[min,max]\\
		
					false & otherwise\\
				\end{array} \right.$\\
		
		
		
		
			$\mathbb{P}_{\pmotif,\ptext,min,max}$ is closed under induction. 
			it can be
			solved by means of the following relations:\\
			
			
			\textbf{BASE :} \\
			
			if $|\pmotif| \notin  \x : $\\		
			$\mathbb{P}_{\bmotif,\ptext,min,max}(|\pmotif|,j)= \left\{ 
					\begin{array}{l l}
						true & \text{if $min<\ptext(j)<max$
						}\\
						false & otherwise\\
					\end{array} \right. $\\	
			$\mathbb{P}_{\bmotif,\ptext,min,max}(|\pmotif|+1,j)= true $\\
			
			if $|\pmotif| \in  \x : $\\		
			$\mathbb{P}_{\bmotif,\ptext,min,max}(|\pmotif|,|\ptext|)= \left\{ 
					\begin{array}{l l}
						true & \text{if $min<\ptext(|\ptext|)<max$
						}\\
						false & otherwise\\
					\end{array} \right. $\\	
			$\mathbb{P}_{\bmotif,\ptext,min,max}(|\pmotif|+1,|\ptext|)= true $\\
			
			if $|\pmotif| \notin  \x$  
			and $|\pmotif| \in  \y$ 
			and $\pmotif(|\pmotif|)$ is the maximal element: \\		
			$\mathbb{P}_{\bmotif,\ptext,min,max}(|\pmotif|,j)= \left\{ 
					\begin{array}{l l}
						true & \text{if $min<\ptext(j)<max$
						}\\
						&\text{$\ptext(j)$ is the maximal element}\\
						false & otherwise\\
					\end{array} \right. $\\	
			$\mathbb{P}_{\bmotif,\ptext,min,max}(|\pmotif|+1,j)= true $\\
			
			if $|\pmotif| \in  \x$  
			and $|\pmotif| \in  \y$ 
			and $\pmotif(|\pmotif|)$ is the maximal element: \\		
			$\mathbb{P}_{\bmotif,\ptext,min,max}(|\pmotif|,|\ptext|)= \left\{ 
					\begin{array}{l l}
						true & \text{if $min<\ptext(|\ptext|)<max$
						}\\
						&\text{$\ptext(|\ptext|)$ is the maximal element}\\
						false & otherwise\\
					\end{array} \right. $\\	
			$\mathbb{P}_{\bmotif,\ptext,min,max}(|\pmotif|+1,|\ptext|)= true $\\		
								
			\textbf{STEP :} \\		
			$\mathbb{P}_{\bmotif,\ptext,min,max}(i,j)= \\ \left\{ 
					\begin{array}{l l}
					
						\bigcup_{j<k}\mathbb{P}_{\bmotif,\ptext,t(j),max}(i+1,k) 
							& \text{if $\mathcal{S}(\pmotif,i)=\ustep$
							and $i \notin \x$}\\
							& \text{and $i \notin \y$ }\\
							& \text{and $min<\ptext(j)<max$} \\
						&\\
							
						\bigcup_{j<k}\mathbb{P}_{\bmotif,\ptext,min,t(j)}(i+1,k) 
							& \text{if $\mathcal{S}(\pmotif,i)=\dstep$}\\
							& \text{and $i \notin \x$ and $i \notin \y$  }\\
							& \text{and $min<\ptext(j)<max$} \\		
						&\\
							
						\mathbb{P}_{\bmotif,\ptext,t(j),max}(i+1,j+1) 
							& \text{if $\mathcal{S}(\pmotif,i)=\ustep$}\\
							& \text{and $i \in \x$ and $i \notin \y$}\\
							& \text{and $min<\ptext(j)<max$} \\
						&\\
							
						\mathbb{P}_{\bmotif,\ptext,min,t(j)}(i+1,j+1) 
							& \text{if $\mathcal{S}(\pmotif,i)=\dstep$}\\
							& \text{and $i \in \x$ and $i \notin \y$ }\\
							& \text{and $min<\ptext(j)<max$} \\
														
						&\\
													
						\mathbb{P}_{\bmotif,\ptext,t(j),max}(i+1,\ptext^{-1}(\ptext(j)+1)) 
							&\text{if $\mathcal{S}(\pmotif,i)=\ustep$}\\
							&\text{and $i \notin \x$ and $i \in \y$}\\
							& \text{and $min<\ptext(j)<max$} \\
							&\text{and $\ptext^{-1}(\ptext(j)+1)>j$}\\
						&\\
							
						\mathbb{P}_{\bmotif,\ptext,min,t(j)}(i+1,\ptext^{-1}(\ptext(j)-1)) 
							&\text{if $\mathcal{S}(\pmotif,i)=\dstep$}\\
							&\text{and $i \notin \x$ and $i \in \y$ }\\	
							& \text{and $min<\ptext(j)<max$} \\
							&\text{and $\ptext^{-1}(\ptext(j)-1)>j$}\\
						&\\
						
						\mathbb{P}_{\bmotif,\ptext,t(j),max}(i+1,j+1) 
							&\text{if $\mathcal{S}(\pmotif,i)=\ustep$}\\
							&\text{and $i \in \x$ and $i \in \y$}\\ 
							&\text{and $\ptext(j)+1=\ptext(j+1)$ }\\
							& \text{and $min<\ptext(j)<max$} \\
						&\\
							
						\mathbb{P}_{\bmotif,\ptext,min,t(j)}(i+1,j+1)
							& \text{if $\mathcal{S}(\pmotif,i)=\dstep$}\\
							&\text{and $i \in \x$ and $i \in \y$}\\ 
							&\text{and $\ptext(j)-1=\ptext(j+1)$ }\\
							& \text{and $min<\ptext(j)<max$} \\
						&\\	
						False & Otherwise\\																		
					\end{array} \right. $\\				
				
%			$\mathbb{P}_{\bmotif,\ptext,min,max}(i,j)= \\ \left\{ 
%					\begin{array}{l l}
%						False & \text{if $\ptext(j)<min$ or $max<\ptext(j)$}\\
%						&\\
%					
%						\bigcup_{j<k}\mathbb{P}_{\bmotif,\ptext,t(j),max}(i+1,k) 
%							& \text{if $\mathcal{S}(\pmotif,i)=\ustep$
%							and $i \notin \x$}\\
%							& \text{and $i \notin \y$ }\\
%						&\\
%							
%						\bigcup_{j<k}\mathbb{P}_{\bmotif,\ptext,min,t(j)}(i+1,k) 
%							& \text{if $\mathcal{S}(\pmotif,i)=\dstep$}\\
%							& \text{and $i \notin \x$ and $i \notin \y$  
%							}\\
%						&\\
%							
%						\mathbb{P}_{\bmotif,\ptext,t(j),max}(i+1,j+1) 
%							& \text{if $\mathcal{S}(\pmotif,i)=\ustep$}\\
%							& \text{and $i \in \x$ and $i \notin \y$  
%							}\\
%						&\\
%							
%						\mathbb{P}_{\bmotif,\ptext,min,t(j)}(i+1,j+1) 
%							& \text{if $\mathcal{S}(\pmotif,i)=\dstep$}\\
%							& \text{and $i \in \x$ and $i \notin \y$  
%							}\\							
%						&\\
%							
%						False 
%						& \text{if $\ptext^{-1}(\ptext(j)+1)<j$}\\
%						&\text{and $\mathcal{S}(\pmotif,i)=\ustep$}\\
%						&\text{and $i \in \y$}\\
%						&\\
%							
%						False 
%						& \text{if $\ptext^{-1}(\ptext(j)-1)<j$}\\
%						& \text{and $\mathcal{S}(\pmotif,i)=\dstep$}\\
%						& \text{and $i \in \y$}\\	
%						&\\
%						
%						\mathbb{P}_{\bmotif,\ptext,t(j),max}(i+1,\ptext^{-1}(\ptext(j)+1)) 
%							& \text{if $\mathcal{S}(\pmotif,i)=\ustep$}\\
%							&\text{and $i \notin \x$ and $i \in \y$ 
%							}\\
%						&\\
%							
%						\mathbb{P}_{\bmotif,\ptext,min,t(j)}(i+1,\ptext^{-1}(\ptext(j)-1)) 
%							& \text{if $\mathcal{S}(\pmotif,i)=\dstep$}\\
%							&\text{and $i \notin \x$ and $i \in \y$  
%							}\\	
%						&\\
%						
%						\mathbb{P}_{\bmotif,\ptext,t(j),max}(i+1,j+1) 
%							& \text{if $\mathcal{S}(\pmotif,i)=\ustep$}\\
%							&\text{and $i \in \x$ and $i \in \y$}\\ 
%							&\text{and $\ptext(j)+1=\ptext(j+1)$ 
%							}\\
%						&\\
%							
%						\mathbb{P}_{\bmotif,\ptext,min,t(j)}(i+1,j+1)
%							& \text{if $\mathcal{S}(\pmotif,i)=\dstep$}\\
%							&\text{and $i \in \x$ and $i \in \y$}\\ 
%							&\text{and $\ptext(j)-1=\ptext(j+1)$ 
%							}\\		
%						&\\	
%						False & Otherwise\\																		
%					\end{array} \right. $\\	
						
			At each step (i,j),
			If $i \notin \x$ and $i \notin \y$,
			we match the current element of $\pmotif$ ($\pmotif(i)$)
			with the current element of $\ptext$ ($\ptext(j)$), if possible.
			Then we match $\pmotif[i+1:]$ with 
			every suffixes $\ptext$ starting after $j$.\\
			
			If $i \in \x$ and $i \notin \y$,
			we match $\pmotif(i)$ to $\ptext(j)$, 
			then we match $\pmotif[i+1:]$ with $\ptext[j+1:]$
			with the condition that
			$\pmotif(i+1)$ is matched to $\ptext(j+1)$.\\
			
			If $i \notin \x$, $i \in \y$ and i is a down step,
			we match $\pmotif(i)$ to $\ptext(j)$, 
			then we match $\pmotif[i+1:]$ 
			with $\ptext[\ptext^{-1}(\ptext(j)-1):]$
			with the condition that
			$\pmotif(i+1)$ is matched to the element $\ptext(j)-1$.\\						
			
			If $i \notin \x$, $i \in \y$ and i is an up step,
			we match $\pmotif(i)$ to $\ptext(j)$, 
			then we match $\pmotif[i+1:]$ 
			with $\ptext[\ptext^{-1}(\ptext(j)+1):]$
			with the condition that
			$\pmotif(i+1)$ is matched to the element $\ptext(j)+1$.\\	

			If $i \in \x$, $i \in \y$ and i is a down step,
			we match $\pmotif(i)$ to $\ptext(j)$, 
			then we match $\pmotif[i+1:]$ with $\ptext[j+1:]$
			with the condition that
			$\pmotif(i+1)$ is matched to $\ptext(j+1)$,
			and that $\ptext(j+1)=\pmotif(i)-1$.\\			

			If $i \in \x$, $i \in \y$ and i is an up step,
			we match $\pmotif(i)$ to $\ptext(j)$, 
			then we match $\pmotif[i+1:]$ with $\ptext[j+1:]$
			with the condition that
			$\pmotif(i+1)$ is matched to $\ptext(j+1)$,
			and that $\ptext(j+1)=\pmotif(i)+1$.\\		
					
			There are two cases to consider searching for a match : \\
			If $|\pmotif| \in \x$ there exists a match
			if and only if\\ $\bigcup_{0<min<max<|\ptext|}\mathbb{P}_{\pmotif,\ptext,min,t(j)}(0,0)$ is true.\\
			
			If $|\pmotif| \notin \x$ there exists a match if and only if \\	$\bigcup_{0<k<|\ptext|,0<min<max<|\ptext|}\mathbb{P}_{\pmotif,\ptext,min,t(j)}(0,k)$
			is true. $\Box$  			
	
	\section{Longest Subsequence Avoiding \\(231,213) for a Permutation}

	In this section we present an algorithm to solve the problem
	of the longest subsequence avoiding (231,213).
	To do so, we need the set of element
	that are matched to an up step and the
	set of element that are matched to a down step.
	For a permutation $p$ of size $n$, we define
	$P(p) = \{i | \mathcal{S}(p,i) = \ustep \} \cup \{n\}$ and  
	$M(p) = \{i | \mathcal{S}(p,i) = \dstep \} \cup \{n\}$.\\
	
	\textbf{Proposition  \numl{longestIncreasingSubsequence} \thenum}
	If $s$ is a longest subsequence avoiding (213,231) with last element at index $f$ in $p$ then
	$P(p)$ is a longest increasing subsequence with last element at index $f$ and
	$M(p)$ is a longest decreasing subsequence with last element at index $f$.\\

	
	\textbf{Proof} Let's $s$ is a longest subsequence avoiding (213,231) with last element at index $f$ in $p$, 
	suppose that $P(p)$ is not a longest increasing subsequence with last element at index $f$. Let's $s_m$ be a longest increasing subsequence with last element $f$.
	Thus $|s_m|>P(p)$, clearly the sequence $s_m \cup M(p)$
	is avoiding (213,231) and is longer than $s$ witch is not possible. 
	The case for $M(p)$ follows the same idea.$\Box$\\

	\textbf{Proposition  \num \thenum}
	Given a permutation $p$, 
	finding the longest subsequence avoiding (231,213)
	can be done in $O(|\ptext|*log(log(|\ptext|)))$ time and in $O(n)$ space.\\
	
	\textbf{Proof} The proposition \ref{longestIncreasingSubsequence} lead to algorithm 
	where we have to compute longest increasing and decreasing subsequence ending at every index. Then finding the maximum sum of longest increasing and decreasing subsequence ending at the same index.
	Computing the longest increasing and decreasing can be done in $O(|\ptext|*log(log(|\ptext|)))$ time and $O(n)$ space (see \cite{Bespamyatnikh00enumeratinglongest}), then finding the maximum can be done in linear time.$\Box$\\



	\section{Longest Subsequence Avoiding \\(231,213) Common for Two Permutation}
	
	In this section we present an algorithm to find 
	the longest common subsequence avoiding (231,213) 
	between two permutations.\\
	
	\textbf{Proposition \num \thenum } Given two permutation $p_1$ and $p_2$
	The longest common subsequence
	avoiding (231,213) can be solve in $O(|p_1|^3*|p_2|^3)$.\\
	
	\textbf{Proof } Consider the following problem, 
	that compute the longest stripe common to $p_1$ and $p_2$.
	Given two permutation $p_1$ and $p_2$.\\
	
	$S_{p_1,p_2}(min_1,max_1,min_2,min_2,i_1,i_2)$
	= max$\{$ $|m|$ $|$ 
	s is a pattern occurring 
	in  $p_1[i_1:]$
	by the subsequence $s_1$ and $min(s_1)=min_1$ and $max(s_1)=max_1$
	and s is occurring
	in  $p_2[i_2:]$
	by the subsequence $s_2$ and $min(s_2)=min_2$ and $max(s_2)=max_2$	
	$\}$\\
	
	We show that this family of problems are closed under induction.\\


	\textbf{BASE :} \\
	$S_{p_1,p_2,min_1,max_1,min_2,max_2}(|p_1|,|p_2|) = \left\{ 
			\begin{array}{l l}
				1 & \text{if $min_1<p_1(j)<max_1$  
				}\\
				& \text{ and $min_2<p_2(j)<max_2$}\\
				0 & otherwise\\
			\end{array} \right. $\\	

	\textbf{STEP :} \\	
	
	$S_{p_1,p_2,min_1,max_1,min_2,max_2}(i_1,i_2) =$\\
	\indent $ max \left\{ 
			\begin{array}{l l}
				S_{p_1,p_2,min_1,max_1,min_2,max_2}(i_1,i_2+1) \\
				S_{p_1,p_2,min_1,max_1,min_2,max_22}(i_1+1,i_2) \\
				S'_{p_1,p_2,min_1,max_1,min_2,max_2}(i_1,i_2)
			\end{array} \right. $\vspace{20pt}\\	
	
	with 
	$S'_{p_1,p_2,min_1,max_1,min_2,max_2}(i_1,i_2)=\\ \left\{ 
				\begin{array}{l l}
					1+S_{p_1,p_2,p_1(i_1),max_1,p_2(i_2)}(max_2,i_1+1,i_2+1)  & \text{$p_1(i_1)<min_1$ } \\
					& \text{and $p_2(i_2)<min_2$} \\
					&\\
					1+S_{p_1,p_2,min_1,p_1(i_1),min_2,p_2(i_2)}(i_1+1,i_2+1)  & \text{$p_1(i_1)>max_1$ } \\	
					&\text{and $p_2(i_2)>max_2$}\\	
					&\\	
					0 & \text{otherwise}\\							
				\end{array} \right.$\\
				
	For every pair $i,j$ we either ignore the element of $p_1$,
	either ignore the element of $p_2$, 
	either we match as the same step (if possible).
	Those relation lead to a $O(|p_1|^3*|p_2|^3)$ time and $O(|p_1|^3*|p_2|^3)$ space algorithm.$\Box$ 		



			
%	\section{Related problems}
%		
%		\textbf{Proposition 3.1} Longest Minimal Common Supersequence, Shortest Maximal Common Subsequence, and Shortest Maximal Common Non-Supersequence are MAX SNP-hard.\\
%		
%		Clearly every problem over (231,213)-Avoiding permutation
%		are equivalent to problem over binary word.
%		The proposition follow what we know about those problem over binary alphabet
%		\cite{Middendorf1995317}.\\
	
	
		
	\section{Conclusion}
		


		  					
\bibliography{bibli}{}
\bibliographystyle{plain}	
			
				      
\end{document}
